\pagestyle{plain}
\begin{abstract}

  ผมวิเคราะห์แยกแยะโครงการโอเพนซอร์สที่ประสบความสำเร็จโครงการหนึ่ง คือ
  fetchmail
  ซึ่งดำเนินการโดยเจตนาจะทดสอบทฤษฎีที่น่าประหลาดใจเกี่ยวกับวิศวกรรมซอฟต์แวร์
  ที่ได้มาจากการพิจารณาความเป็นมาของลินุกซ์
  ผมกล่าวถึงทฤษฎีเหล่านี้ในมุมมองของรูปแบบการพัฒนาสองแนวทางที่แตกต่างกันโดยสิ้นเชิง
  คือรูปแบบ ``มหาวิหาร'' ที่ใช้กันในโลกพาณิชย์เกือบทั้งหมด กับรูปแบบ
  ``ตลาดสด'' ของโลกลินุกซ์ ผมแสดงให้เห็นว่า
  รูปแบบเหล่านี้เกิดจากข้อสมมุติที่ขัดแย้งกัน
  เกี่ยวกับธรรมชาติของงานแก้บั๊กซอฟต์แวร์ จากนั้น
  ผมได้ให้ทัศนะพร้อมเหตุผลรองรับที่ได้จากประสบการณ์ของลินุกซ์สำหรับข้อเสนอที่ว่า
  ``ขอให้มีสายตาเฝ้ามองมากพอ บั๊กทั้งหมดก็เป็นเรื่องง่าย''
  ผมเปรียบข้อเสนอดังกล่าวกับระบบซึ่งมีการแก้ไขตัวเองของตัวกระทำที่เห็นแก่ตัว
  และสรุปด้วยการสำรวจนัยของแนวคิดนี้สำหรับอนาคตของวงการซอฟต์แวร์
\end{abstract}