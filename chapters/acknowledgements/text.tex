\chapter{กิติกรรมประกาศ}

บทความนี้ได้รับการปรับปรุงด้วยการสนทนากับผู้คนจำนวนมากที่ช่วยตรวจทาน
ขอขอบคุณ Jeff Dutky \href{mailto:dutky@wam.umd.edu}{dutky@wam.umd.edu}
ซึ่งได้แนะนำคำสรุปที่ว่า ``การแก้บั๊กสามารถทำขนานกันได้''
และได้ช่วยวิเคราะห์ตามคำสรุปดังกล่าวด้วย ขอบคุณ Nancy Lebovitz \href{mailto:nancyl@universe.digex.net}{nancyl@universe.digex.net}
สำหรับคำแนะนำว่า ผมได้เลียนแบบเวนเบิร์กด้วยการอ้างคำพูดของโครพอตกิน
มีคำวิจารณ์ที่ลึกซึ้งจาก Joan Eslinger
\href{mailto:wombat@kilimanjaro.engr.sgi.com}{wombat@{\wbr}kilimanjaro.engr.sgi.com}  และ
Marty Franz \href{mailto:marty@net-link.net}{marty@net-link.net}
จากเมลลิ่งลิสต์ General Technics นอกจากนี้ Glen Vandenburg \href{mailto:glv@vanderburg.org}{glv@vanderburg.org}
ยังได้ชี้ให้เห็นถึงความสำคัญของการกลั่นกรองตัวเองของประชากรผู้สมทบ
และแนะนำแนวคิดที่ช่วยให้เกิดผลอย่างมาก
ว่างานพัฒนาปริมาณมากถือว่าเป็นการแก้ `บั๊กเนื่องจากสิ่งที่ขาดไป' Daniel
Upper \href{mailto:upper@peak.org}{upper@peak.org}
ได้แนะนำการเปรียบเทียบกับธรรมชาติของสิ่งนี้ ผมรู้สึกขอบคุณต่อสมาชิกของ
PLUG หรือ Philadelphia Linux User's Group
ที่ได้หาผู้ทดลองอ่านชุดแรกสำหรับบทความรุ่นแรก Paula Matuszek
\href{mailto:matusp00@mh.us.sbphrd.com}{matusp00@mh.us.{\wbr}sbphrd.com}
ได้ให้ความกระจ่างแก่ผมเกี่ยวกับวิธีการบริหารงานซอฟต์แวร์ Phil Hudson
\href{mailto:phil.hudson@iname.com}{phil.hudson@iname.com} เตือนผมว่า
การจัดโครงสร้างของวัฒนธรรมแฮ็กเกอร์ จะสะท้อนโครงสร้างของซอฟต์แวร์
และในทางกลับกันก็เป็นจริงด้วย John Buck
\href{mailto:johnbuck@sea.ece.umassd.edu}{johnbuck{\wbr}@sea.ece.umassd.edu}  ชี้ว่า
MATLAB ก็ให้ตัวอย่างที่เหมือนกับ Emacs และ Russell Johnston
\href{mailto:russjj@mail.com}{russjj@mail.com}
ทำให้ผมได้สติเกี่ยวกับกลไกบางอย่างที่อภิปรายในหัวข้อ
``สายตากี่คู่ที่จะจัดการกับความซับซ้อนได้'' และท้ายที่สุด ความเห็นของ
ไลนัส ทอร์วัลด์ เป็นประโยชน์มาก และการให้ความเห็นชอบของเขาตั้งแต่เนิ่น ๆ
นั้น ช่วยเป็นกำลังใจได้มาก