\chapter{เมื่อใดที่กุหลาบจะไม่เป็นกุหลาบ?}

เมื่อได้ศึกษาวิธีการของไลนัส และสร้างทฤษฎีว่ามันสำเร็จได้อย่างไรแล้ว
ผมได้ตัดสินใจที่จะทดสอบทฤษฎีนี้กับโครงการของผมเอง
(ที่ต้องยอมรับว่าซับซ้อนน้อยกว่าและทะเยอทะยานน้อยกว่ามาก)

แต่สิ่งแรกที่ผมทำ คือการปรับ popclient ให้ลดความซับซ้อนลง งานของคาร์ล
แฮรริส นั้นดูดี แต่ก็ได้สร้างความซับซ้อนที่ไม่จำเป็น
ตามแบบที่โปรแกรมเมอร์ภาษาซีหลายคนชอบทำ เขามองตัวโค้ดเป็นศูนย์กลาง
และโครงสร้างข้อมูลเป็นส่วนสนับสนุนของโค้ด ผลก็คือ โค้ดนั้นสวยงามดี
แต่โครงสร้างข้อมูลออกแบบตามอำเภอใจและค่อนข้างแย่
(อย่างน้อยก็แย่ตามมาตรฐานที่สูงอยู่แล้วของแฮ็กเกอร์ Lisp มือฉมังคนนี้)

อย่างไรก็ดี ผมมีอีกจุดประสงค์หนึ่งในการเขียนใหม่
นอกจากการปรับปรุงโค้ดและโครงสร้างข้อมูล
นั่นคือการปรับโปรแกรมให้กลายเป็นสิ่งที่ผมเข้าใจได้ทั้งหมด
มันไม่สนุกเลยในการรับผิดชอบแก้บั๊กของโปรแกรมที่คุณไม่เข้าใจ

ในราวเดือนแรก ผมทำตามแนวทางที่คาร์ลออกแบบไว้
ความเปลี่ยนแปลงที่สำคัญอย่างแรกที่ผมทำ คือเพิ่มการสนับสนุน IMAP
ผมทำโดยการจัดระบบโพรโทคอลใหม่ให้เป็นไดรเวอร์ทั่วไปหนึ่งตัว
และตารางวิธีการสามอัน (สำหรับ POP2, POP3 และ IMAP)
การเปลี่ยนแปลงทั้งครั้งนี้และครั้งก่อน
แสดงให้เห็นหลักการทั่วไปที่โปรแกรมเมอร์ควรจำให้ขึ้นใจ
โดยเฉพาะสำหรับภาษาอย่างซี ซึ่งไม่ได้สนับสนุนชนิดข้อมูลแบบพลวัต นั่นคือ:

\begin{fancyquotes}
  9. โครงสร้างข้อมูลที่ฉลาดกับโค้ดที่โง่ ทำงานได้ดีกว่าในทางกลับกัน
\end{fancyquotes}

บรูกส์, บทที่ 9 : ``ให้ผมดูโฟลว์ชาร์ตของคุณ แล้วปิดตารางของคุณไว้
ผมก็ยังจะงงต่อไป แต่ถ้าให้ผมดูตารางของคุณ
ผมแทบไม่ต้อดูโฟลว์ชาร์ตของคุณเลย ทุกอย่างชัดเจน''
เมื่อพิจารณาเวลาสามสิบปีของการเปลี่ยนแปลงของคำศัพท์และยุคสมัยแล้ว
ประเด็นก็ยังเหมือนเดิม

ณ จุดนี้ (ต้นเดือนกันยา 1996 ทำมาแล้วประมาณ 6 อาทิตย์)
ผมเริ่มคิดจะเปลี่ยนชื่อโครงการแล้ว เพราะมันไม่ใช่แค่โปรแกรมสำหรับ POP
อีกต่อไป แต่ผมยังไม่แน่ใจ เพราะมันยังไม่มีอะไรใหม่จริง ๆ  ในด้านการออกแบบ
popclient รุ่นของผมยังไม่ได้พัฒนาเอกลักษณ์เป็นของตัวเองเลย

แต่เงื่อนไขนั้นก็ได้เปลี่ยนไปอย่างมาก เมื่อ popclient
เริ่มจะส่งเมลต่อไปยังพอร์ต SMTP ได้ ผมจะกลับมาพูดถึงเรื่องนี้อีกที
แต่ก่อนอื่น ผมพูดไว้ก่อนหน้านี้ว่า
ผมตั้งใจจะใช้โครงการนี้พิสูจน์ทฤษฎีของผมเกี่ยวกับสิ่งที่ ไลนัส ทอร์วัลด์
ทำ คุณอาจจะถามว่า ผมทำอย่างไรบ้าง? ผมทำอย่างนี้:

\begin{itemize}
  \item
        ผมออกเนิ่น ๆ  และถี่ ๆ  (แทบจะไม่เคยทิ้งช่วงเกินสิบวัน
        และถ้าเป็นช่วงที่มีการพัฒนาแบบเข้มข้น ผมออกทุกวัน)
  \item
        ผมเพิ่มรายชื่อผู้ทดสอบโดยใส่ชื่อทุกคนที่คุยกับผมเรื่อง fetchmail
  \item
        ผมประกาศแบบเป็นกันเองไปยังผู้ทดสอบทุกคนเมื่อออก
        กระตุ้นให้ผู้คนมีส่วนร่วม
  \item
        และผมฟังผู้ทดสอบเบต้า ขอความเห็นเกี่ยวกับการออกแบบ
        และขอบคุณเมื่อเขาส่งแพตช์ และความคิดเห็นมาให้
\end{itemize}

ผลลัพธ์จากมาตรการง่าย ๆ  เหล่านี้เห็นได้รวดเร็ว ตั้งแต่เริ่มโครงการ
ผมได้รับการรายงานบั๊กที่มีคุณภาพระดับที่นักพัฒนาอยากได้ใจจะขาด
และบ่อยครั้งที่มีวิธีแก้มาให้ด้วย ผมได้รับคำวิจารณ์ที่ให้แนวคิดที่ดี
ได้รับเมลจากแฟน ๆ  ได้รับคำแนะนำคุณสมบัติใหม่ ๆ  ซึ่งนำไปสู่:

\begin{fancyquotes}
  10. ถ้าคุณปฏิบัติกับผู้ทดสอบรุ่นเบต้า
  เหมือนกับเป็นแหล่งทรัพยากรชั้นเยี่ยมแล้ว
  เขาจะตอบแทนด้วยการเป็นทรัพยากรชั้นเยี่ยมให้
\end{fancyquotes}

ดัชนีชี้วัดความสำเร็จของ fetchmail ที่น่าสนใจอย่างหนึ่ง
คือขนาดของเมลลิงลิสต์ fetchmail-friends ของผู้ทดสอบเบต้าของโครงการ
ขณะที่แก้ไขปรับปรุงบทความนี้รุ่นล่าสุด (พฤศจิกายน 2000) มีสมาชิกถึง 287
คน และเพิ่มขึ้น 2-3 คนทุกสัปดาห์

ความจริงแล้ว ตั้งแต่ผมเริ่มแก้ไขปรับปรุงบทความในปลายเดือนพฤษภาคม 1997
ผมพบว่าคนในรายชื่อเริ่มจะลดจำนวนลงจากเกือบ 300 คนในตอนแรก
ด้วยเหตุผลที่น่าสนใจ คือ หลาย ๆ  คนได้ขอให้ผมเอาชื่อเขาออก เพราะ fetchmail
ทำงานได้ดีแล้ว จนพวกเขาไม่ต้องการมีส่วนร่วมในการสนทนาอีก บางที
นี่อาจจะเป็นส่วนหนึ่งในวงจรชีวิตของโครงการแบบตลาดสดที่โตเต็มที่แล้วก็ได้