\chapter{มหาวิหารกับตลาดสด}

ลินุกซ์คือผู้ล้มยักษ์ เมื่อ 5 ปีที่แล้ว (ปี 1991)
ใครจะไปคิดว่าระบบปฏิบัติการระดับโลก
จะก่อตัวขึ้นราวกับมีเวทมนตร์จากการแฮ็กเล่น  ๆ
ในเวลาว่างของนักพัฒนานับพันจากทั่วโลกที่เชื่อมต่อกันด้วยเส้นใยบาง  ๆ
อย่างอินเทอร์เน็ตเท่านั้น

ผมคนหนึ่งล่ะ ที่ไม่เชื่อ
ตอนที่ลินุกซ์เข้ามาอยู่ในความสนใจของผมเมื่อต้นปี 1993 นั้น
ผมได้เข้ามาเกี่ยวข้องกับยูนิกซ์ และการพัฒนาแบบโอเพนซอร์สมาสิบปีแล้ว
ผมยังเป็นหนึ่งในผู้สมทบงานให้ GNU เป็นคนแรก  ๆ   ในช่วงกลางทศวรรษ 1980
ผมได้ปล่อยซอฟต์แวร์โอเพนซอร์สออกสู่อินเทอร์เน็ตแล้วหลายตัว
โดยได้สร้างและร่วมสร้างโปรแกรมหลายโปรแกรม (nethack, โหมด VC และ GUD ของ
Emacs, xlife และอื่น  ๆ  ) ซึ่งยังคงใช้กันอยู่แพร่หลายในทุกวันนี้
ผมคิดว่าตัวเองรู้ดีเรื่องการพัฒนาซอฟต์แวร์

แต่ลินุกซ์ได้ลบล้างสิ่งที่ผมเคยคิดว่ารู้ไปมาก
ผมเคยพร่ำสอนเกี่ยวกับบัญญัติยูนิกซ์ เรื่องการเขียนโปรแกรมขนาดเล็ก
การสร้างต้นแบบอย่างเร็ว และการเขียนโปรแกรมแบบวิวัฒนาการมาหลายปี
แต่ผมยังเชื่ออีกด้วย ว่ามีความซับซ้อนวิกฤติระดับหนึ่ง ที่ถ้าเลยขั้นนี้ไป
ก็ต้องใช้วิธีพัฒนาที่รวมศูนย์ มีทฤษฎีมากกว่านั้น
ผมเชื่อว่าซอฟต์แวร์ที่สำคัญ  ๆ   (เช่น ระบบปฏิบัติการ
และโปรแกรมขนาดใหญ่อย่าง Emacs) ควรจะถูกสร้างเหมือนสร้างมหาวิหาร
(cathedral) โดยพ่อมดซอฟต์แวร์สักคน หรือผู้วิเศษกลุ่มเล็ก  ๆ
เป็นผู้ประดิษฐ์ขึ้นอย่างบรรจง ในดินแดนโดดเดี่ยวอันศักดิ์สิทธิ์
ไม่มีตัวทดสอบ (beta) ออกมาให้ลองก่อนเวอร์ชันจริง

วิธีการพัฒนาของไลนัส ทอร์วัลด์ เป็นเรื่องแปลกประหลาด วิธีของเขาคือ
`ออกเนิ่น  ๆ   ออกถี่  ๆ   มอบงานทุกส่วนให้คนอื่นเท่าที่จะทำได้
และเปิดกว้างถึงขั้นสำส่อน'
นี่ไม่ใช่การสร้างมหาวิหารอย่างเงียบเชียบด้วยความเทิดทูนบูชา
ชุมชนของลินุกซ์นั้น เหมือนกับตลาดสด (bazaar)
ที่เอะอะอื้ออึงฟังไม่ได้ศัพท์
ซึ่งแต่ละคนมีวาระและวิธีการที่แตกต่างหลากหลาย (เห็นได้จากไซต์ FTP
ของลินุกซ์ ที่ใครก็สามารถส่งผลงานของตัวเองเข้ามาได้)
การจะเกิดระบบปฏิบัติการที่เสถียรและเป็นเอกภาพขึ้นได้จากสภาพดังกล่าว
จึงดูเหมือนต้องเป็นผลจากปาฏิหาริย์เท่านั้น

ความจริงที่ว่าการพัฒนาแบบตลาดสดนี้ใช้งานได้ และได้ผลดีด้วยนั้น
เป็นเรื่องน่าตกใจมาก ขณะที่ผมเรียนรู้ไปเรื่อย  ๆ   นั้น
ผมไม่เพียงทุ่มเทให้กับโครงการทั้งหลาย แต่ผมยังพยายามหาสาเหตุ
ว่าทำไมโลกของลินุกซ์จึงไม่เพียงไม่แตกเป็นเสี่ยง  ๆ   ด้วยความโกลาหล
แต่ยังกลับแข็งแกร่งและมั่นคงขึ้นเรื่อย  ๆ
ด้วยอัตราเร็วที่นักสร้างมหาวิหารแทบไม่สามารถจินตนาการถึงได้

กลางปี 1996 ผมคิดว่าผมเริ่มเข้าใจแล้ว
ผมมีโอกาสอันยอดเยี่ยมที่จะทดสอบทฤษฎีของตัวเอง
ในรูปแบบของโครงการโอเพนซอร์ส ซึ่งผมสามารถเจาะจงให้พัฒนาในแบบตลาดสดได้
ผมจึงลองทำดู และมันก็ประสบความสำเร็จดีทีเดียว

เรื่องราวต่อไปนี้เป็นเรื่องของโครงการดังกล่าว
ผมจะใช้ตัวอย่างนี้เสนอคติสำหรับการพัฒนาแบบโอเพนซอร์สที่ได้ผล
หลายอย่างไม่ใช่สิ่งที่ผมเพิ่งเรียนรู้เป็นครั้งแรกจากโลกของลินุกซ์
แต่เราจะเห็นว่าโลกของลินุกซ์ทำให้มันสำคัญขึ้นมาอย่างไร ถ้าผมคิดไม่ผิด
คติเหล่านี้จะช่วยให้คุณเข้าใจมากขึ้น
ว่าอะไรคือสิ่งที่ทำให้สังคมลินุกซ์กลายเป็นบ่อเกิดของซอฟต์แวร์ดี  ๆ
และอาจช่วยทำให้คุณพัฒนาผลิตภาพของคุณเองให้มากขึ้นได้ด้วย