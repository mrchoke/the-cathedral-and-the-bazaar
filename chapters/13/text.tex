\chapter{ส่งท้าย: เน็ตสเคปอ้าแขนรับตลาดสด}

เป็นความรู้สึกที่แปลกที่ได้รู้ว่า คุณกำลังร่วมสร้างประวัติศาสตร์...

วันที่ 22 มกราคม 1998 ประมาณเจ็ดเดือนหลังจากที่ผมเผยแพร่บทความ
\emph{มหาวิหารกับตลาดสด} บริษัท เน็ตสเคป คอมมิวนิเคชัน ได้ประกาศแผนที่จะ
\href{http://www.netscape.com/newsref/pr/newsrelease558.html}{แจกซอร์สของ
  Netscape Communicator} ผมไม่เคยคาดคิดมาก่อนเลย ว่าสิ่งนี้จะเกิดขึ้น
จนถึงวันประกาศ

Eric Hahn รองประธานบริหารและหัวหน้าฝ่ายเทคโนโลยีของเน็ตสเคป
ได้ส่งอีเมลสั้น ๆ  ถึงผมในภายหลัง บอกว่า: ``ในนามของทุก ๆ  คนที่เน็ตสเคป
ผมอยากจะขอบคุณที่คุณได้ช่วยเราตั้งแต่ต้นให้มาถึงจุดนี้
ความคิดและข้อเขียนของคุณคือแรงบันดาลใจโดยพื้นฐานของการตัดสินใจของเรา''

ในสัปดาห์ต่อมา ผมบินไปที่ซิลิคอนแวลลีย์ตามคำเชิญของเน็ตสเคป
เพื่อเข้าร่วมงานประชุมระยะเวลาหนึ่งวัน (ในวันที่ 4 ก.พ. 1998)
กับผู้บริหารระดับสูงและผู้เชี่ยวชาญทางเทคนิคของบริษัท
เราร่วมกันวางกลยุทธ์การปล่อยซอร์ส และสัญญาอนุญาตของเน็ตสเคป

\noindent สองสามวันถัดมา ผมเขียนว่า:

\begin{fancyquotes}
  เน็ตสเคปกำลังจะจัดเตรียมการทดสอบจริงขนาดใหญ่
  สำหรับรูปแบบตลาด\-สดในโลกธุรกิจ ขณะนี้
  โลกโอเพนซอร์สกำลังเผชิญกับอันตราย
  ถ้าการดำเนินการของเน็ตสเคปไม่เป็นผล
  แนวคิดโอเพนซอร์สอาจจะถูกลดความน่าเชื่อถือลง
  ถึงขนาดที่จะไม่ได้รับความสนใจจากโลกธุรกิจอีกเลยเป็นสิบปี

  ในทางกลับกัน นี่ก็ถือเป็นโอกาสอันงดงามเช่นกัน
  ปฏิกิริยาเบื้องต้นต่อความเคลื่อนไหวครั้งนี้ในวอลล์สตรีทและที่อื่น ๆ
  กำลังเป็นบวกอย่างระมัดระวัง เรากำลังได้รับโอกาสพิสูจน์ตัวเองด้วย
  ถ้าเน็ตสเคปสามารถรุกส่วนแบ่งตลาดกลับมาได้จากความเคลื่อนไหวครั้งนี้
  ก็อาจเป็นการเริ่มต้นปฏิวัติวงการอุตสาหกรรมซอฟต์แวร์
  ซึ่งควรจะเกิดมาตั้งนานแล้ว

  ปีหน้านี้ น่าจะเป็นช่วงเวลาที่น่าศึกษาและน่าสนใจอย่างยิ่ง
\end{fancyquotes}

และมันก็เป็นเช่นนั้นจริง ๆ  ขณะที่ผมเขียนเรื่องนี้ในกลางปี 2000
การพัฒนาของสิ่งที่ได้รับการขนานนามต่อมาว่า โมซิลล่า
ได้กลายเป็นความสำเร็จระดับคุณภาพ
โดยสามารถบรรลุเป้าหมายเริ่มแรกของเน็ตสเคป
คือการปฏิเสธการผูกขาดถาวรของไมโครซอฟท์ในตลาดเบราว์เซอร์
และยังประสบความสำเร็จอย่างถล่มทลายอีกด้วย
(โดยเฉพาะการออกเครื่องจักรซอฟต์แวร์สำหรับวาดหน้าเว็บรุ่นใหม่ที่ชื่อ
เก็คโค)

อย่างไรก็ดี มันก็ยังไม่ได้รับแรงพัฒนาอย่างใหญ่หลวงจากภายนอกเน็ตสเคป
อย่างที่ผู้ก่อตั้งโมซิลล่าคาดหวังไว้แต่ต้น ปัญหาของที่นี่
ดูจะเป็นเพราะการแจกจ่ายโมซิลล่าได้แหกกฎพื้นฐานของรูปแบบตลาดสดเป็นเวลานาน
กล่าวคือ
ไม่ได้ให้สิ่งที่ผู้ที่อาจร่วมสมทบสามารถเรียกใช้เพื่อดูการทำงานได้
(ตลอดเวลากว่าหนึ่งปีตั้งแต่ออก
การคอมไพล์โมซิลล่าจากซอร์สจำเป็นต้องอาศัยสิทธิ์อนุญาตใช้งานสำหรับไลบรารีโมทีฟ
ซึ่งเป็นซอฟต์แวร์สงวนสิทธิ์)

ที่เป็นลบมากที่สุด (จากมุมมองของโลกภายนอก)
คือกลุ่มโมซิลล่าไม่ได้ให้เบราว์เซอร์ระดับคุณภาพเป็นเวลาถึงสองปีครึ่งหลังจากตั้งโครงการ
และในปี 1999 แกนนำคนสำคัญของโครงการก็ได้สร้างความรู้สึกที่ไม่ดี
ด้วยการลาออก และบ่นถึงการบริหารที่แย่ และการเสียโอกาสต่าง ๆ
``โอเพนซอร์ส'' เขาตั้งข้อสังเกตอย่างถูกต้อง ``ไม่ใช่เวทมนตร์วิเศษ''

และมันก็ไม่ใช่จริง ๆ  อาการต่าง ๆ
ในระยะยาวของโมซิลล่าดูจะดีขึ้นอย่างมากในขณะนี้ (ในเดือนพฤศจิกายน 2000)
เทียบกับขณะที่ เจมี่ ซาวินสกี้ เขียนจดหมายลาออก ในช่วงอาทิตย์หลัง ๆ
รุ่นปล่อยประจำวันได้ผ่านระดับคุณภาพที่สำคัญไปสู่การใช้งานจริงเป็นที่เรียบร้อย
แต่เจมี่ก็กล่าวได้ถูกต้อง ที่ชี้ให้เห็นว่า
การเปิดซอร์สไม่จำเป็นต้องช่วยชีวิตโครงการเดิมที่กำหนดเป้าหมายไว้แย่ ๆ
หรือมีโค้ดที่ยุ่งเหยิง หรือมีอาการป่วยเรื้อรังอื่น ๆ
ทางวิศวกรรมซอฟต์แวร์ได้เสมอไป โมซิลล่าได้แสดงตัวอย่าง
ทั้งการประสบความสำเร็จและการล้มเหลวของโอเพนซอร์สในเวลาเดียวกัน

อย่างไรก็ดี ในช่วงเวลาเดียวกันนี้ แนวคิดโอเพนซอร์สก็ได้ประสบความสำเร็จ
และมีผู้สนับสนุนในที่อื่น ๆ  ตั้งแต่การออกซอฟต์แวร์ของเน็ตสเคป
เราก็ได้เห็นการบูมอย่างมโหฬารของความสนใจในรูปแบบการพัฒนาแบบโอเพนซอร์ส
ซึ่งเป็นแนวโน้มที่ทั้งรับและให้แรงขับเคลื่อนต่อความสำเร็จของระบบปฏิบัติการลินุกซ์ในเวลาเดียวกัน
แนวโน้มที่โมซิลล่าได้จุดประกายขึ้น
ได้ดำเนินต่อไปในอัตราที่สูงขึ้นเรื่อย ๆ
