\chapter{จาก Popclient สู่ Fetchmail}

จุดเปลี่ยนที่แท้จริงของโครงการ เกิดขึ้นเมื่อ Harry Hochheiser
ส่งร่างโค้ดของเขาสำหรับการส่งเมลต่อไปยังพอร์ต SMTP
ของเครื่องลูกข่ายมาให้ผม
ผมรู้ทันทีว่าความสามารถนี้ถ้าทำงานได้จริงอย่างเชื่อถือได้แล้ว
จะทำให้วิธีกระจายเมลแบบอื่น ๆ  เตรียมม้วนเสื่อไปได้เลย

เป็นเวลาหลายสัปดาห์ ที่ผมได้ปรับปรุงต่อเติม fetchmail
โดยรู้สึกว่าส่วนติดต่อนั้นทำงานได้ดี แต่ดูรกรุงรัง
ไม่สวยและมีตัวเลือกหยุมหยิมเต็มไปหมด
ผมรำคาญตัวเลือกที่ให้โยนเมลที่ดึงมาไปลงแฟ้ม mailbox หรือเอาต์พุตมาตรฐาน
แต่ก็ไม่รู้เหมือนกันว่าทำไม

\noindent (ถ้าคุณไม่สนเรื่องทางเทคนิคของการส่งเมลในอินเทอร์เน็ต
ก็อ่านข้ามสองย่อหน้าถัดไปนี้ได้เลย)

สิ่งที่ผมเห็นเมื่อคิดถึงการส่งเมลต่อไปยัง SMTP ก็คือ pop\-client %
นั้นพยายามทำงานหลายอย่างเกินไป มันถูกออกแบบให้เป็นทั้งโปรแกรมจัดส่งเมลภายนอก
(mail transport agent -- MTA) และโปรแกรมกระจายเมลภายในเครื่อง (local delivery agent -- MDA)
เมื่อมีความสามารถในการส่งเมลต่อไปยัง SMTP แล้ว มันก็ควรจะเลิกทำตัวเป็น
MDA และเป็นเพียง MTA เพียงอย่างเดียว
แล้วโอนหน้าที่ในการกระจายเมลในเครื่องไปให้โปรแกรมอื่น เหมือนกับที่
sendmail ทำอยู่

ทำไมต้องไปยุ่งกับรายละเอียดในการตั้งค่าการกระจายเมล
หรือการล็อคกล่องเมลก่อนเขียนต่อท้าย ในเมื่อแทบจะแน่ใจได้ว่ามีพอร์ต 25
ให้ใช้ในทุกแพล็ตฟอร์มที่สนับสนุน TCP/IP อยู่แล้ว? โดยเฉพาะอย่างยิ่ง
เมื่อการใช้พอร์ตดังกล่าวยังรับประกันได้ว่าจะทำให้เมลที่ดึงมานั้น
ดูเหมือนเมล SMTP ปกติที่รับมาจากผู้ส่งโดยตรง
ซึ่งเป็นสิ่งที่เราต้องการจริง ๆ  อยู่แล้ว

\noindent (กลับสู่เรื่องเดิม...)

ถึงแม้คุณจะไม่ได้ติดตามศัพท์แสงทางเทคนิคในย่อหน้าก่อน
แต่ก็ยังมีบทเรียนที่สำคัญหลายบทสำหรับเรื่องนี้ สิ่งแรกก็คือ
การส่งเมลต่อไปยัง SMTP
เป็นผลลัพธ์ที่ดีที่สุดที่ผมได้รับจากการเลียนแบบวิธีพัฒนาของไลนัสอย่างจงใจ
ผู้ใช้คนหนึ่งได้ให้แนวคิดสุดยอดอันนี้
สิ่งที่ผมต้องทำคือเข้าใจความหมายและสิ่งที่จะตามมา

\begin{fancyquotes}
  11. สิ่งที่ดีที่สุดรองจากการมีแนวคิดดี ๆ
  ก็คือการตระหนักถึงแนวคิดที่ดีจากผู้ใช้ของคุณ บางครั้ง
  การตระหนักดังกล่าวก็ถือว่าสำคัญกว่า
\end{fancyquotes}

สิ่งที่น่าสนใจตามมาคือ คุณจะค้นพบอย่างรวดเร็ว
ว่าถ้าคุณซื่อสัตย์กับการบอกว่าได้แนวคิดมาจากผู้ใช้มากเท่าไร
โลกภายนอกยิ่งจะมองว่าคุณเป็นคนสร้างสิ่งนั้นขึ้นมาเองทุกกระเบียดนิ้ว
และมองคุณว่าเป็นอัจฉริยะที่ถ่อมตัว ดูอย่างไลนัสสิ!

(ตอนที่ผมพูดบนเวทีในงาน Perl Conference เดือนสิงหาคมปี 1997 ลาร์รี วอลล์
แฮ็กเกอร์ผู้ยิ่งยงนั่งอยู่แถวหน้า เมื่อผมพูดถึงเรื่องในย่อหน้าที่ผ่านมา
เขาตะโกนขึ้นราวกับเสียงปลุกเร้าของนักบุญ ว่า "บอกเขาไป บอกเขาไปให้หมด
เพื่อน!" ผู้ฟังทั้งหมดหัวเราะครืน
เพราะรู้ว่าเรื่องนี้เกิดกับเขาซึ่งเป็นผู้สร้างภาษา Perl ด้วย)

สองสามสัปดาห์จากการทำงานโครงการนี้ด้วยแนวคิดเดียวกัน
ผมเริ่มได้รับการยกย่องคล้าย ๆ  กันนี้ ไม่ใช่แค่จากผู้ใช้ของผม
แต่ยังมาจากคนอื่น ๆ  ที่ได้ยินเรื่องของโครงการด้วย
ผมซุกเมลเหล่านั้นบางฉบับออกไป ผมอาจจะหยิบกลับมาอ่านในบางครั้ง
ถ้าเกิดสงสัยขึ้นมาว่าชีวิตผมมีค่าหรือเปล่า :-)

แต่ยังมีบทเรียนพื้นฐานที่ไม่เกี่ยวกับการเมืองอีกสองข้อ
ที่เกี่ยวกับการออกแบบทุกชนิดโดยทั่วไป

\begin{fancyquotes}
  12. บ่อยครั้งที่วิธีการที่เฉียบแหลมและแปลกใหม่ จะมาจากการตระหนักว่า
  คุณมองปัญหานั้นผิดมาตลอด
\end{fancyquotes}

ผมเคยพยายามแก้ปัญหาผิดประเด็น โดยการพัฒนา popclient ให้เป็นทั้ง MTA/MDA
ในตัวเดียวกันต่อไป พร้อมกับโหมดการกระจายเมลแบบต่าง ๆ  ภายในเครื่อง
แต่การออกแบบ fetchmail ต้องเริ่มคิดใหม่แต่ต้นให้เป็น MTA ล้วน ๆ
โดยเป็นส่วนหนึ่งของเส้นทางเมลในอินเทอร์เน็ตผ่านโพรโทคอล SMTP

เมื่อคุณเจอทางตันในการพัฒนา เมื่อคุณพบว่าตัวเองต้องคิดหนักกับแพตช์ถัดไป
ก็มักจะเป็นเวลาที่จะเลิกถามตัวเองว่า ``ได้คำตอบที่ถูกต้องหรือยัง''
แต่ควรจะถามใหม่ว่า ``ตั้งคำถามถูกหรือเปล่า''
บางทีก็อาจต้องนิยามปัญหาใหม่

เอาล่ะ ผมได้มองปัญหาของผมใหม่ ชัดเจนว่าสิ่งที่ควรทำคือ (1)
เพิ่มการส่งเมลต่อไปยัง SMTP เข้าไปในไดรเวอร์ทั่วไป (2)
ทำให้มันเป็นโหมดปริยาย และ (3) เอาโหมดอื่น ๆ  ออกไปในที่สุด โดยเฉพาะ
โหมดกระจายไปยังแฟ้ม (deliver-to-file) และโหมดกระจายไปยังเอาต์พุตมาตรฐาน
(deliver-to-standard-output)

ผมลังเลที่จะทำขั้นที่ 3 อยู่ระยะหนึ่ง เพราะกลัวว่าจะทำให้ผู้ที่ใช้
popclient มานานที่อาจจะใช้วิธีกระจายเมลแบบอื่นอยู่ไม่พอใจ ตามทฤษฎีแล้ว
เขาสามารถเปลี่ยนไปแก้แฟ้ม \texttt{.forward} (หรือแฟ้มอื่น ๆ
ที่เทียบเท่าถ้าไม่ใช้ sendmail) ได้ทันที โดยได้ผลเหมือนเดิม
แต่ในทางปฏิบัติ การเปลี่ยนดังกล่าวอาจจะยุ่งยาก

แต่เมื่อผมได้ทำจริง ๆ  ผลที่ได้นั้นดีมาก
ส่วนที่ยุ่งที่สุดของไดรเวอร์ถูกตัดออกไป
การปรับแต่งค่าทำได้ง่ายขึ้นอย่างเห็นได้ชัด ไม่จำเป็นต้องไปยุ่งกับทั้ง
MDA และกล่องเมลของผู้ใช้อีกต่อไป ไม่ต้องกังวลว่า OS
ที่ใช้อยู่สนับสนุนการล็อคแฟ้มหรือเปล่า

นอกจากนั้น โอกาสเดียวที่จะทำเมลหายก็หมดไปอีกด้วย เดิมที
ถ้าคุณระบุให้กระจายไปยังแฟ้ม (delivery-to-file)
และเนื้อที่ดิสก์เกิดเต็มขึ้นมา เมลนั้นจะหายไป
แต่สิ่งนี้จะไม่เกิดขึ้นกับการส่งต่อไปยัง SMTP เพราะว่าโปรแกรมรับเมลแบบ
SMTP จะไม่ยอมตกลงจนกว่าเมลจะถูกกระจายไปเรียบร้อย
หรืออย่างน้อยก็ส่งเข้าที่พักไว้ รอการกระจายต่อในภายหลัง

เรื่องประสิทธิภาพก็ยังสูงขึ้นอีกด้วย
(แม้จะไม่รู้สึกในการทำงานครั้งเดียว) ผลดีอีกอย่างที่ไม่สำคัญเท่าไร
คือคู่มือวิธีใช้ (man page) ดูง่ายลงมาก

ต่อมาภายหลัง ผมต้องนำส่วนกระจายเมลผ่าน MDA ที่ผู้ใช้กำหนดกลับมาอีก
เพื่อจัดการกับบางสถานการณ์ที่เกี่ยวกับ SLIP
แต่ผมพบวิธีที่ง่ายกว่าเดิมมากในการเพิ่มมันเข้าไป

คติของเรื่องนี้น่ะหรือ? อย่าลังเลที่จะทิ้งความสามารถที่หมดอายุแล้ว
เมื่อคุณพบว่าคุณสามารถทำได้โดยผลลัพธ์ยังเท่าเดิม อังตวน เดอ
แซง-เตกซูเปรี (ผู้เป็นนักบินและนักออกแบบเครื่องบิน
ในช่วงที่ไม่ได้เขียนหนังสืออมตะสำหรับเด็ก) กล่าวไว้ว่า:

\begin{fancyquotes}
  13. ``ความสมบูรณ์แบบ (ในการออกแบบ) จะได้มาไม่ใช่เมื่อไม่มีอะไรจะเพิ่ม
  แต่จะได้มาเมื่อไม่มีอะไรจะเอาออกต่างหาก''
\end{fancyquotes}

เมื่อโค้ดขของคุณดีขึ้นและเรียบง่ายขึ้น เมื่อนั้นแหละที่คุณจะรู้สึกว่ามัน
{[}\emph{ใช่}{]} และในกระบวนการนี้ การออกแบบของ fetchmail
ได้สร้างเอกลักษณ์ของตัวเอง ซึ่งแตกต่างไปจาก popclient เดิม

มันถึงเวลาที่จะเปลี่ยนชื่อซะที การออกแบบแบบใหม่ดูเหมือนกับเป็นคู่ของ
send\-mail มากกว่าที่ popclient เดิมเคยเป็น โปรแกรมทั้งคู่เป็น MTA
แต่ในขณะที่ sendmail จะผลักออกไปแล้วกระจาย แต่ popclient
ตัวใหม่จะดึงเข้ามาแล้วกระจาย สองเดือนถัดมา ผมเปลี่ยนชื่อมันเป็น
fetchmail

มีบทเรียนทั่ว ๆ  ไปอีกข้อจากเรื่องเกี่ยวกับวิธีที่การกระจายเมลแบบ SMTP
กลายมาเป็น fetchmail นี้ คือบทเรียนที่ว่า ไม่ใช่แค่การตรวจบั๊กเท่านั้น
ที่ทำขนานกันได้
แต่การพัฒนาและการสำรวจความเป็นไปได้ของการออกแบบก็ทำขนานได้
(ในระดับที่น่าประหลาดใจ) เช่นกัน
เมื่อการพัฒนาของคุณมีรูปแบบวนรอบอย่างรวดเร็ว
การพัฒนาและเพิ่มความสามารถจะกลายเป็นการแก้บั๊กกรณีพิเศษ นั่นคือ
`บั๊กเนื่องจากสิ่งที่ขาดไป' ในคุณสมบัติหรือแนวคิดเริ่มแรกของตัวซอฟต์แวร์

แม้กับการออกแบบระดับบน
การมีผู้ร่วมพัฒนาจำนวนมากเดินผ่านการออกแบบของคุณในทิศทางต่าง ๆ
ก็ยังมีประโยชน์มาก ๆ  ลองคิดถึงการที่ก้อนน้ำหาทางไหลจนลงท่อ
หรือมดที่หาอาหาร ต่างเป็นการสำรวจโดยใช้การแพร่กระจาย
ตามด้วยการช่วงใช้ที่จัดการผ่านกลไกการสื่อสาร วิธีนี้ใช้การได้ดีมาก
เหมือนกับที่ผมและ Harry Hochheiser ทำ ใครบางคนจากภายนอก
อาจพบสิ่งสุดยอดที่อยู่ใกล้ ๆ  ตัวคุณ ที่คุณอยู่ใกล้เกินกว่าจะมองเห็นก็ได้
