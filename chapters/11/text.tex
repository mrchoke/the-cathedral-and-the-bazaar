\chapter{สภาพแวดล้อมทางสังคมของซอฟต์แวร์โอเพนซอร์ส}

มีข้อเขียนเกี่ยวกับเรื่องนี้จริง ๆ  ว่าการแฮ็กที่ดีที่สุด
เริ่มจากวิธีส่วนตัวในการแก้ปัญหาที่ตัวผู้เขียนพบในชีวิตประจำวัน
และแพร่หลายออกไป เพราะปรากฏว่าเป็นปัญหาปกติที่ผู้ใช้จำนวนมากก็พบเช่นกัน
เรื่องนี้นำเรากลับไปสู่กฏข้อที่ 1
ซึ่งกล่าวใหม่ในแบบที่มีประโยชน์ขึ้นได้ว่า:

\begin{fancyquotes}
  18. การแก้ปัญหาที่น่าสนใจ เริ่มจากการค้นหาปัญหาที่คุณสนใจ
\end{fancyquotes}

ซึ่งเหมือนที่เกิดกับ คาร์ล แฮร์ริส กับ popclient รุ่นแรก ๆ  และกับผมกับ
fetchmail อย่างไรก็ดี เรื่องนี้เป็นที่เข้าใจกันมานานแล้ว
แต่จุดที่น่าสนใจจริง ๆ  โดยเฉพาะในประวัติของลินุกซ์และ fetchmail
ที่เราต้องศึกษา ก็คือขั้นต่อไปต่างหาก กล่าวคือ
วิวัฒนาการของซอฟต์แวร์ในชุมชนผู้ใช้และผู้ร่วมพัฒนาที่มีขนาดใหญ่และตื่นตัวสูง

ในหนังสือ \emph{The Mythical Man-Month} เฟรด บรูกส์ ตั้งข้อสังเกตว่า
เวลาในการทำงานของโปรแกรมเมอร์นั้นทดแทนกันไม่ได้
การเพิ่มนักพัฒนาให้กับโครงการซอฟต์แวร์ที่ล่าช้าอยู่
จะทำให้มันยิ่งล่าช้ายิ่งขึ้น ดังที่เราได้กล่าวไปแล้ว
เขาอ้างว่าความซับซ้อนและค่าโสหุ้ยในการสื่อสารของโครงการ
จะเพิ่มขึ้นในอัตรากำลังสองของจำนวนนักพัฒนา ในขณะที่งานที่ได้
จะเพิ่มขึ้นเป็นลักษณะเส้นตรงเท่านั้น
กฎของบรูกส์ได้รับการยอมรับโดยทั่วไปว่าเป็นความจริง
แต่เราได้วิเคราะห์มาแล้วในบทความนี้ ถึงวิธีต่าง ๆ
ที่กระบวนการพัฒนาแบบโอเพนซอร์สได้ทะลายข้อสมมุติต่าง ๆ  ของกฎนี้
และถ้าว่ากันที่ผลในทางปฏิบัติ หากกฎของบรูกส์ครอบคลุมทั้งหมดแล้ว
ลินุกซ์ก็คือสิ่งที่เป็นไปไม่ได้

หนังสืออมตะของ เจอรัลด์ เวนเบิร์ก ชื่อ
\emph{จิตวิทยาของการเขียนโปรแกรมคอมพิวเตอร์ (The Psychology of Computer
  Programming)}
เสนอสิ่งที่เราเข้าใจเบื้องหลังได้ว่าเป็นการแก้ไขกฎของบรูกส์ครั้งสำคัญ
ในการอภิปรายเกี่ยวกับ ``การเขียนโปรแกรมแบบไร้อัตตา''
เวนเบิร์กตั้งข้อสังเกตว่า ในหน่วยงานที่นักพัฒนาไม่มีการหวงห้ามโค้ด
และยังเชิญชวนให้คนอื่น ๆ  ให้มาช่วยกันหาข้อผิดพลาด
และจุดที่อาจจะพัฒนาต่อได้ จะมีการพัฒนาได้เร็วกว่าหน่วยงานอื่น ๆ
อย่างเห็นได้ชัด (เร็ว ๆ  นี้ เทคนิค `extreme programming' ของ เคนต์ เบ็ก
ที่ให้นักพัฒนาจับคู่กันดูโค้ดกันและกัน
อาจเป็นความพยายามหนึ่งที่จะบังคับให้เกิดผลดังกล่าว)

บางที การเลือกใช้คำของเวนเบิร์ก
อาจทำให้การวิเคราะห์ของเขาไม่ได้รับการยอมรับเท่าที่ควรจะเป็น
บางคนอาจจะอมยิ้มเมื่อนึกถึงการบรรยายถึงแฮ็กเกอร์ในอินเทอร์เน็ตด้วยคำว่า
``ไร้อัตตา'' แต่ผมคิดว่าเหตุผลของเขานั้นดีเหลือเกิน
โดยเฉพาะอย่างยิ่งในปัจจุบัน

ด้วยการใช้ประโยชน์จากผลของ ``การเขียนโปรแกรมแบบไร้อัตตา'' อย่างเต็มพิกัด
วิธีการแบบตลาดสดได้บรรเทาผลของกฎของบรูกส์ลงอย่างมาก
มันไม่ถึงกับทำลายหลักการเบื้องหลังของกฎของบรูกส์ลง
แต่ด้วยจำนวนนักพัฒนาที่มากพอ และด้วยการสื่อสารที่สะดวก
ผลของมันสามารถถูกกลบได้ด้วยปัจจัยตรงข้ามที่ไม่ได้มีลักษณะเป็นเชิงเส้น
ซึ่งจะมองไม่เห็นจนกว่าจะเกิดเงื่อนไขดังกล่าวขึ้น เรื่องนี้
ก็คล้ายกับความสัมพันธ์ระหว่างฟิสิกส์แบบนิวตันและแบบไอน์สไตน์ กล่าวคือ
ระบบเก่าก็ยังใช้การได้ที่ระดับพลังงานต่ำ
แต่ถ้าคุณเพิ่มมวลและความเร็วมากพอ
คุณก็จะพบเรื่องแปลกประหลาดอย่างการระเบิดแบบนิวเคลียร์ หรือลินุกซ์

ประวัติของยูนิกซ์ควรจะช่วยปูพื้นให้กับสิ่งที่เราเรียนรู้จากลินุกซ์ได้
(รวมทั้งสิ่งที่ผมได้ลงมือตรวจสอบผ่านการทดลองในขนาดที่เล็กลง
ด้วยการเลียนแบบวิธีของไลนัสอย่างเจตนา \pagenote{ขณะนี้
  เรามีประวัติศาสตร์ของโครงการที่อาจเป็นการทดสอบที่ชี้วัดศักยภาพของรูปแบบตลาดสดได้ดีกว่า
  fetchmail ในหลาย ๆ  ทาง คือ \href{http://egcs.cygnus.com/}{EGCS}
  (Experimental GNU Compiler System)

  โครงการนี้ประกาศตัวเมื่อกลางเดือนสิงหาคม 1997 โดยเป็นความพยายามอย่างจงใจ
  ที่จะใช้แนวคิดจากบทความ \emph{มหาวิหารกับตลาดสด} ที่เผยแพร่รุ่นแรก
  ผู้ก่อตั้งโครงการรู้สึกว่า การพัฒนาของ GCC หรือ GNU C Compiler
  กำลังติดขัด เป็นเวลายี่สิบเดือนตั้งแต่นั้น ที่ GCC และ EGCS
  กลายเป็นผลิตภัณฑ์ที่คู่ขนานกัน
  โดยดึงแรงงานจากนักพัฒนาในอินเทอร์เน็ตกลุ่มเดียวกัน เริ่มต้นจากซอร์สของ
  GCC เดียวกัน ใช้ชุดเครื่องมือและสภาพแวดล้อมการพัฒนาของยูนิกซ์เหมือนกัน
  แต่ต่างกันตรงที่ EGCS พยายามใช้เทคนิคของตลาดสดที่ผมได้บรรยายไปแล้ว
  ในขณะที่ GCC ยังคงใช้โครงสร้างการทำงานที่คล้ายมหาวิหารมากกว่า
  และทำโดยกลุ่มนักพัฒนาที่ปิด และออกรุ่นไม่บ่อย

  นี่ถือว่าใกล้เคียงกับการทดลองที่ควบคุมตัวแปรมากที่สุดเท่าที่จะทำได้
  และผลลัพธ์ก็เห็นได้อย่างรวดเร็ว ภายในไม่กี่เดือน EGCS
  ได้พัฒนาความสามารถไปไกลกว่าอย่างเห็นได้ชัด ทั้งออปติไมซ์ได้ดีกว่า
  และสนับสนุนภาษาฟอร์แทรนและซีพลัสพลัสได้ดีกว่า หลายคนพบว่า EGCS
  รุ่นระหว่างพัฒนายังเชื่อถือได้กว่ารุ่นเสถียรล่าสุดของ GCC เสียอีก
  และดิสทริบิวชันลินุกซ์ต่าง ๆ  ก็เริ่มจะเปลี่ยนมาใช้ EGCS แทน

  ในเดือนเมษายน 1999 มูลนิธิซอฟต์แวร์เสรี (ผู้สนับสนุนอย่างเป็นทางการของ
  GCC) ได้ยุบกลุ่มพัฒนา GCC เดิมเสีย
  แล้วโอนการควบคุมของโครงการไปให้ทีมหลักของ EGCS แทน} นั่นคือ
ในขณะที่โดยเนื้อแท้แล้วการเขียนโค้ดยังคงเป็นกิจกรรมที่ต้องทำคนเดียวอยู่
แต่การแฮ็กที่สุดยอดจริง ๆ
กลับเกิดจากการควบคุมทิศทางความสนใจและพลังสมองของชุมชนทั้งหมด
นักพัฒนาที่ใช้แค่สมองของตนเองในโครงการปิด
กำลังจะถูกแซงโดยนักพัฒนาที่รู้วิธีสร้างสภาพแวดล้อมแบบเปิดเพื่อการวิวัฒน์
ที่ซึ่งผลตอบรับที่ช่วยสำรวจวิธีออกแบบที่เป็นไปได้ การสมทบโค้ด
การชี้ข้อผิดพลาด และการปรับปรุงอื่น ๆ  จะมาจากคนเป็นร้อย ๆ
(หรืออาจเป็นพัน ๆ )

แต่โลกของยูนิกซ์สมัยก่อน กลับไม่สามารถใช้วิธีนี้ไปถึงจุดสูงสุดได้
ด้วยปัจจัยหลายประการ
ปัจจัยหนึ่งก็คือข้อจำกัดทางกฎหมายของสัญญาอนุญาตแบบต่าง ๆ  ความลับทางการค้า
และความสนใจทางธุรกิจ นอกจากนี้ อีกปัจจัยหนึ่งที่เข้าใจได้
คืออินเทอร์เน็ตยังใช้การได้ไม่ดีพอในตอนนั้น

ก่อนที่อินเทอร์เน็ตจะมีราคาถูกอย่างทุกวันนี้ ได้มีชุมชนเล็ก ๆ
ที่รวมตัวกันด้วยพื้นที่ทางภูมิศาสตร์บางกลุ่ม
ซึ่งมีวัฒนธรรมที่สนับสนุนให้เกิดการเขียนโปรแกรมแบบ ``ไร้อัตตา''
ของเวนเบิร์ก
และนักพัฒนาสามารถดึงดูดผู้อยากรู้อยากเห็นและผู้ร่วมพัฒนาที่มีทักษะจำนวนมากได้อย่างง่ายดาย
ชุมชนอย่าง เบลล์แล็บ, แล็บ AI และ LCS ของ MIT, UC Berkeley
เหล่านี้ได้กลายเป็นบ่อเกิดของนวัตกรรมระดับตำนานมากมาย
และทุกวันนี้ชุมชนเหล่านี้ก็ยังคงแสดงศักยภาพอยู่

ลินุกซ์เป็นโครงการแรกที่สามารถใช้ {[}\emph{โลกทั้งใบ}{]}
เป็นแหล่งของพรสวรรค์ได้อย่างจงใจและประสบความสำเร็จ
ผมไม่คิดว่ามันเป็นเรื่องบังเอิญ ที่ช่วงเวลาบ่มเพาะตัวของลินุกซ์
พอดีกันกับการเกิดของเครือข่ายใยแมงมุม (World Wide Web)
และที่ลินุกซ์เริ่มเติบโตในระหว่างปี 1993--1994
ซึ่งตรงกับช่วงที่เกิดอุตสาหกรรมการให้บริการอินเทอร์เน็ต
และเกิดการบูมของอินเทอร์เน็ตในความสนใจกระแสหลักของผู้คน
ไลนัสเป็นคนแรกที่รู้วิธีเล่นกับกฎเกณฑ์ใหม่ ๆ  ที่อินเทอร์เน็ตได้สร้างขึ้น

แม้อินเทอร์เน็ตราคาถูก จะเป็นเงื่อนไขที่จำเป็นในการพัฒนาในแบบลินุกซ์
แต่ผมคิดว่าแค่สิ่งนี้อย่างเดียวยังไม่เพียงพอ
ปัจจัยที่สำคัญอีกอย่างหนึ่งก็คือ การพัฒนาของรูปแบบความเป็นผู้นำ
และประเพณีแห่งความร่วมมือต่าง ๆ  ซึ่งทำให้ผู้พัฒนาสามารถดึงดูดผู้ร่วมพัฒนา
และใช้ประโยชน์สูงสุดจากสื่ออย่างอินเทอร์เน็ต

แต่เป็นรูปแบบความเป็นผู้นำแบบไหน? และอะไรคือประเพณีต่าง ๆ  ที่ว่า?
สิ่งเหล่านี้ไม่สามารถได้มาจากการใช้อำนาจ หรือถึงแม้ว่ามันจะใช้ได้
แต่การเป็นผู้นำที่ใช้การบังคับ
ก็ไม่สามารถสร้างผลลัพธ์อย่างที่เราเห็นอยู่ทุกวันนี้ได้
เวนเบิร์กยกคำกล่าวจากอัตชีวประวัติของ ปโยต์ อเล็กเซวิช โครพอตกิน (Pyotr
Alexeyvich Kropotkin) อนาธิปัตย์ชาวรัสเซียในคริสต์ศตวรรษที่ 19 ในหนังสือ
\emph{ความทรงจำของนักปฏิวัติ (Memoirs of a Revolutionist)}
ซึ่งเกี่ยวข้องกับเรื่องนี้อย่างมาก:

\begin{fancyquotes}
  ด้วยความที่เติบโตมาในครอบครัวที่มีทาส
  ผมได้เข้าสู่ชีวิตที่กระฉับกระเฉงเหมือน ๆ  กับชายหนุ่มอื่น ๆ  ในยุคของผม
  ซึ่งมีความมั่นใจในความจำเป็นของการบังคับบัญชา การออกคำสั่ง การตำหนิ
  การลงโทษ และอะไรทำนองนี้ แต่เมื่อผมได้เริ่มบริหารบรรษัทอย่างจริงจัง
  และต้องติดต่อจัดการกับผู้คนซึ่งไม่ใช่ทาส ทั้งความผิดพลาดแต่ละครั้ง
  อาจจะนำไปสู่ความเสียหายอันใหญ่หลวง
  ผมก็ได้เริ่มตระหนักถึงความแตกต่างระหว่าง
  หลักแห่งการบังคับบัญชาและระเบียบวินัย กับ
  หลักแห่งการทำความเข้าใจขั้นพื้นฐานร่วมกัน
  หลักอันแรกใช้การได้ดีกับการเดินสวนสนามของทหาร
  แต่ไม่มีประโยชน์อะไรเลยในชีวิตจริง จุดหมายที่ตั้งไว้จะบรรลุได้
  ก็ด้วยความทุ่มเทอย่างแข็งขัน
  จากเจตน์จำนงอันหลากหลายที่มีเป้าหมายร่วมกันเท่านั้น
\end{fancyquotes}

``ความทุ่มเทอย่างแข็งขัน จากเจตน์จำนงอันหลากหลายที่มีเป้าหมายร่วมกัน''
คือสิ่งที่โครงการอย่างลินุกซ์ต้องการอย่างไม่ต้องสงสัย และไม่มีทางที่
``หลักแห่งการบังคับบัญชา''
จะใช้ได้ผลกับเหล่าอาสาสมัครในสวรรค์ของอนาธิปัตย์ที่เราเรียกว่าอินเทอร์เน็ต
การที่จะดำเนินงานและแข่งขันอย่างได้ผล
เหล่าแฮ็กเกอร์ที่ต้องการจะเป็นผู้นำโครงการที่ต้องการความร่วมมือใด ๆ
จะต้องเรียนรู้วิธีที่จะสรรหาและกระตุ้นชุมชนที่สนใจจริง
ในแบบที่กล่าวไว้อย่างหลวม ๆ  ใน ``หลักแห่งความเข้าใจกัน'' ของโครพอตกิน
พวกเขาต้องเรียนรู้ที่จะใช้กฏของไลนัส \pagenote{แน่นอน คำวิจารณ์ของโครพอตกินและกฎของไลนัส
  ได้สร้างประเด็นกว้าง ๆ  เกี่ยวกับกลไกไซเบอร์สำหรับจัดโครงสร้างสังคม
  ทฤษฎีชาวบ้านอีกทฤษฎีหนึ่งของวิศวกรรมซอฟต์แวร์
  ก็ได้ชี้ให้เห็นอีกประเด็นหนึ่ง คือกฎของคอนเวย์ ซึ่งกล่าวกันโดยทั่วไปว่า
  ``ถ้าคุณมีทีมงานสี่ทีมร่วมกันทำคอมไพเลอร์
  คุณก็จะได้คอมไพเลอร์ที่ทำงานสี่ขั้น''
  ข้อความดั้งเดิมอยู่ในรูปทั่วไปกว่านั้น: ``องค์กรต่าง ๆ  ที่ออกแบบระบบ
  จะถูกบังคับให้สร้างระบบที่สะท้อนโครงสร้างการสื่อสารขององค์กรเหล่านั้น''
  เราอาจกล่าวอย่างย่นย่อกว่านั้นได้ว่า ``วิธีการจะกำหนดผลลัพธ์''
  หรือแม้แต่ว่า ``กระบวนการจะกลายเป็นผลิตภัณฑ์''

  น่าสังเกตพอ ๆ  กัน ว่าในชุมชนโอเพนซอร์สนั้น
  รูปแบบโครงสร้างชุมชนก็ตรงกับหน้าที่ที่ทำในหลายระดับ
  เครือข่ายนี้ครอบคลุมทุกอย่างและทุกที่ ไม่ใช่แค่อินเทอร์เน็ต
  แต่ผู้คนที่ทำงานยังได้สร้างเครือข่ายแบบกระจาย ขึ้นต่อกันอย่างหลวม ๆ
  ในระดับเดียวกัน ที่มีส่วนที่ทดแทนกันได้เกิดขึ้นกลายส่วน
  และไม่ล้มครืนลงแบบทันทีทันใด ในเครือข่ายทั้งสอง
  แต่ละกลุ่มจะมีความสำคัญแค่ในระดับที่กลุ่มอื่นต้องการจะร่วมมือด้วยเท่านั้น

  ตรงส่วน ``ในระดับเดียวกัน'' นี้ สำคัญมากสำหรับผลิตภาพอันน่าทึ่งของชุมชน
  ประเด็นที่โครพอตกินพยายามจะชี้เกี่ยวกับความสัมพันธ์เชิงอำนาจ
  ได้ถูกพัฒนาต่อไปโดย `หลัก SNAFU' ที่ว่า ``การสื่อสารที่แท้จริง
  จะเกิดได้ระหว่างคนที่เท่าเทียมกันเท่านั้น
  เพราะผู้ที่ด้อยกว่าจะได้รับการตอบแทนอย่างสม่ำเสมอกว่า
  ถ้าพูดโกหกให้ผู้ที่เหนือกว่าพอใจ เทียบกับการพูดความจริง''
  ทีมงานที่สร้างสรรค์จะขึ้นอยู่กับการสื่อสารอย่างแท้จริง
  และจะถูกขัดขวางอย่างมากจากการมีความสัมพันธ์เชิงอำนาจ ชุมชนโอเพนซอร์ส
  ซึ่งปราศจากความสัมพันธ์เชิงอำนาจดังกล่าว จึงได้สอนเราในทางตรงกันข้าม
  ให้รู้ถึงข้อเสียของความสัมพันธ์ดังกล่าวในรูปของบั๊ก ผลิตภาพที่ถดถอย
  และโอกาสที่สูญเสียไป

  นอกจากนี้ หลัก SNAFU ยังได้ทำนายว่า ในองค์กรที่มีอำนาจหน้าที่นั้น
  จะเกิดการตัดขาดระหว่างผู้มีอำนาจตัดสินใจ กับความเป็นจริง
  เพราะข้อมูลที่ผู้มีอำนาจตัดสินใจจะได้รับ
  มักมีแนวโน้มจะเป็นการโกหกให้พอใจ
  การเกิดเหตุการณ์เช่นนี้ในการพัฒนาซอฟต์แวร์แบบเดิม ก็เข้าใจได้ง่าย
  เนื่องจากมีแรงจูงใจอย่างแรงกล้าสำหรับผู้ที่ด้อยกว่า ที่จะซ่อน เพิกเฉย
  และลดปัญหาลง เมื่อกระบวนการนี้กลายมาเป็นผลิตภัณฑ์
  ซอฟต์แวร์ก็กลายเป็นหายนะ}

ก่อนหน้านี้ ผมได้กล่าวถึง ``ปรากฏการณ์เดลไฟ''
ในฐานะคำอธิบายที่พอจะใช้ได้สำหรับกฎของไลนัส
แต่ระบบที่ปรับตัวได้ในทางชีววิทยาและเศรษฐศาสตร์
ก็น่าจะเป็นตัวเปรียบเทียบที่ดีเช่นกัน โลกของลินุกซ์มีลักษณะหลาย ๆ
อย่างคล้ายคลึงกับตลาดเสรีหรือระบบนิเวศน์
ซึ่งประกอบด้วยกลุ่มของตัวกระทำที่เห็นแก่ตัว
ที่พยายามหาทางที่จะได้ประโยชน์สูงสุด โดยในกระบวนการนั้น
จะเกิดการจัดระเบียบที่มีการแก้ไขตัวเองแบบเป็นไปเอง
ซึ่งจะมีผลแทรกซึมและมีประสิทธิภาพเกินกว่าที่การวางแผนจากส่วนกลางใด ๆ
จะทำได้ และที่นี่เองที่เราจะมองหา ``หลักแห่งความเข้าใจกัน''

``ฟังก์ชันอรรถประโยชน์'' ที่แฮ็กเกอร์ลินุกซ์พยายามจะทำให้ได้มากที่สุด
ไม่ได้เกี่ยวกับเรื่องเศรษฐกิจ
แต่เกี่ยวกับความพึงพอใจส่วนตัวและชื่อเสียงในหมู่แฮ็กเกอร์ด้วยกัน
(บางคนอาจเรียกแรงจูงใจของพวกเขาเหล่านั้นว่า
``ความเห็นแก่ประโยชน์ส่วนรวม'' โดยมองข้ามความจริงที่ว่า
``ความเห็นแก่ประโยชน์ส่วนรวม''
ก็คืออีกรูปแบบหนึ่งของความพึงพอใจส่วนตัวของ
``ผู้ที่เห็นแก่ประโยชน์ส่วนรวม'') จะว่าไปแล้ว
วัฒนธรรมอาสาสมัครในลักษณะนี้ ก็ไม่ใช่เรื่องพิเศษอะไร
มีอีกกลุ่มหนึ่งที่ผมได้เข้าร่วมมานานแล้ว คือกลุ่มผู้รักนิยายวิทยาศาสตร์
กลุ่มนี้ต่างจากกลุ่มแฮกเกอร์ตรงที่พวกเขาตระหนักใน ``อีโก้บู'' (egoboo
มาจาก ego-boosting
หรือการเพิ่มชื่อเสียงของบุคคลในหมู่คนคลั่งไคล้สิ่งเดียวกัน)
อย่างชัดแจ้งมาเป็นเวลานานแล้ว
ในฐานะสิ่งจูงใจพื้นฐานในการเข้าร่วมกิจกรรมอาสาสมัคร

ไลนัส ซึ่งประสบความสำเร็จในการวางตัวเองเป็นผู้ดูแลโครงการ
โดยที่การพัฒนาส่วนใหญ่ทำโดยคนอื่น ๆ
และเฝ้าหล่อเลี้ยงความสนใจในตัวโครงการจนกระทั่งมันอยู่ได้ด้วยตัวเอง
ได้แสดงให้เห็นถึงความเข้าใจอันเฉียบแหลมใน ``หลักแห่งความเข้าใจร่วมกัน''
ของโครพอตกิน มุมมองต่อโลกของลินุกซ์ในลักษณะกึ่งเศรษฐศาสตร์แบบนี้
เปิดโอกาสให้เราได้เห็นว่า ความเข้าใจนั้นถูกนำไปใช้อย่างไร

เราอาจจะมองวิธีของไลนัสเป็นเหมือนการสร้างตลาด ``อีโก้บู''
ที่มีประสิทธิภาพ
โดยร้อยรัดความเห็นแก่ตัวของแฮ็กเกอร์แต่ละคนอย่างแน่นหนาที่สุดเท่าที่จะทำได้
เข้ากับปลายทางอันยากลำบาก
ซึ่งจะไปถึงได้ก็ด้วยความร่วมมืออย่างไม่ลดละเท่านั้น อย่างในโครงการ
fetchmail ผมก็ได้แสดงให้ดู (แม้จะในขนาดที่เล็กกว่า)
ว่าวิธีของเขาสามารถทำซ้ำได้ โดยให้ผลที่ดี บางที
ผมอาจจะทำแบบจงใจและมีแบบแผนกว่าเขานิดหน่อยด้วยซ้ำ

หลายคน (โดยเฉพาะผู้ที่มีความเคลือบแคลงทางการเมืองกับตลาดเสรี)
อาจคาดไว้ว่า วัฒนธรรมของเหล่าคนอัตตาสูงที่เป็นตัวของตัวเอง คงจะแตกกระจาย
แบ่งแยก สิ้นเปลือง ลึกลับ และไม่เป็นมิตร
แต่สิ่งที่คาดไว้นั้นกลับถูกหักล้างอย่างชัดเจนโดยความหลากหลาย คุณภาพ
และความลึกของเนื้อหาของเอกสารลินุกซ์ (หากจะยกมาเพียงตัวอย่างเดียว)
เป็นคำสาปที่รู้กันดี ว่าโปรแกรมเมอร์นั้น {[}\emph{เกลียด}{]}
การเขียนเอกสาร
แล้วแฮ็กเกอร์ลินุกซ์สร้างเอกสารออกมาได้มากมายอย่างนี้ได้อย่างไร?
อย่างที่เราได้เห็น
ตลาดเสรีสำหรับอีโก้บูของลินุกซ์สามารถสร้างพฤติกรรมที่มีคุณค่า
และสนองความต้องการของผู้อื่น
ได้ดีกว่าหน่วยผลิตเอกสารของผู้ผลิตซอฟต์แวร์เชิงพาณิชย์
ที่มีทุนสนับสนุนอย่างมหาศาลเสียอีก

ทั้งโครงการ fetchmail และเคอร์เนลลินุกซ์ ได้แสดงให้เห็นว่า
ด้วยการตอบแทนอัตตาของแฮกเกอร์หลาย ๆ  คน อย่างเหมาะสม
ผู้พัฒนาและผู้ประสานงานที่มีความสามารถ
สามารถใช้อินเทอร์เน็ตดึงเอาข้อดีของการมีผู้ร่วมพัฒนาเยอะ ๆ  ออกมาได้
โดยไม่ทำให้เกิดความสับสนอลหม่าน ดังนั้นเพื่อแย้งกับกฎของบรูกส์
ผมขอเสนอกฎข้อต่อไปนี้:

\begin{fancyquotes}
  19. หากผู้ประสานงานมีสื่อที่ดีอย่างน้อยเท่ากับอินเทอร์เน็ต
  และรู้ว่าจะนำการพัฒนาโดยไม่ต้องบังคับได้อย่างไร
  หลายหัวย่อมดีกว่าหัวเดียวแน่นอน
\end{fancyquotes}

ผมคิดว่า อนาคตของซอฟต์แวร์โอเพนซอร์ส
จะเป็นของผู้ที่รู้วิธีเล่นเกมของไลนัสมากขึ้นเรื่อย ๆ
คนซึ่งหันหลังให้กับมหาวิหาร และอ้าแขนรับตลาดสด นี่ไม่ได้หมายความว่า
วิสัยทัศน์และความหลักแหลมส่วนบุคคลจะไม่มีความหมายอีกต่อไป ถ้าจะพูดให้ถูก
ผมคิดว่า ความรุดหน้าของซอฟต์แวร์โอเพนซอร์ส
จะเป็นของผู้ที่เริ่มจากวิสัยทัศน์และความหลักแหลมเฉพาะบุคคล
และขยายมันออกไปด้วยการสร้างชุมชนอาสาสมัครที่สนใจเรื่องนั้นอย่างเกิดผล

บางที นี่อาจจะไม่ได้เป็นแค่อนาคตของซอฟต์แวร์ {[}\emph{โอเพนซอร์ส}{]}
เท่านั้น ไม่มีผู้พัฒนาแบบซอร์สปิดรายไหน
ที่จะเทียบเคียงได้กับศูนย์รวมของผู้มีพรสวรรค์ซึ่งชุมชนลินุกซ์สามารถใช้รับมือกับปัญหา
และมีน้อยรายมาก ที่จะสามารถจ้างคนได้มากกว่า 200 คน (600 ในปี 1999, 800
ในปี 2000) อย่างกลุ่มคนที่ช่วยพัฒนา fetchmail!

บางที ที่สุดแล้ว วัฒนธรรมโอเพนซอร์สจะประสบชัยชนะ
ไม่ใช่เพราะความร่วมมือเป็นสิ่งที่ชอบด้วยศีลธรรม หรือ การ ``ปิดบัง''
ซอฟต์แวร์เป็นเรื่องผิดศีลธรรม (ในกรณีที่คุณเชื่อในอย่างหลัง
ซึ่งทั้งไลนัสและผมไม่เชื่อ) แต่เพียงเพราะว่า
โลกของการปิดซอร์สไม่สามารถเอาชนะการแข่งขันทางวิวัฒนาการกับชุมชนโอเพนซอร์ส
ซึ่งสามารถระดมเวลาอันเปี่ยมไปด้วยทักษะจำนวนมาก มาให้กับปัญหาหนึ่ง ๆ  ได้