\chapter{เงื่อนไขตั้งต้นที่จำเป็นสำหรับแนวทางตลาดสด}

ผู้ตรวจทานและผู้ทดลองอ่านบทความนี้คนแรก ๆ  ต่างตั้งคำถามเหมือน ๆ  กัน
เกี่ยวกับเงื่อนไขตั้งต้นที่จำเป็นสำหรับการพัฒนาแบบตลาดสด
รวมทั้งคุณสมบัติของผู้นำโครงการ
และสถานะของโค้ดขณะที่ออกสู่สาธารณะเพื่อเริ่มสร้างชุมชนผู้ร่วมพัฒนา

\noindent ค่อนข้างชัดเจนอยู่แล้ว
ว่าไม่มีใครสามารถเขียนโค้ดตั้งแต่ต้นในแบบตลาดสดได้ \pagenote{ประเด็นที่เกี่ยวข้องกับเรื่องที่นักพัฒนาจะสามารถตั้งต้นโครงการจากศูนย์ในแบบตลาดสดได้หรือไม่
ก็คือประเด็นว่า
รูปแบบตลาดสดสามารถสนับสนุนงานที่เป็นนวัตกรรมอย่างแท้จริงได้หรือไม่
บางคนอ้างว่า หากปราศจากความเป็นผู้นำที่เข้มแข็งแล้ว
ตลาดสดก็สามารถทำได้เพียงจัดการการลอกเลียนและปรับปรุงแนวคิดเดิมที่มีอยู่ที่อยู่ในขั้นประดิษฐ์คิดค้นเท่านั้น
คำโต้แย้งที่เป็นที่รู้จักกันมากที่สุด คงเป็น
\href{http://www.opensource.org/halloween/}{เอกสารวันฮัลโลวีน}
ซึ่งเป็นบันทึกข้อความสองชิ้นที่น่ากระอักกระอ่วนของไมโครซอฟท์
ที่เขียนเกี่ยวกับปรากฏการณ์โอเพนซอร์ส
ผู้เขียนเอกสารได้เปรียบเทียบการพัฒนาระบบปฏิบัติการที่คล้ายยูนิกซ์ของลินุกซ์กับการ
``ไล่กวดไฟท้าย'' และแสดงความเห็นว่า ``(เมื่อโครงการได้ประสบความสำเร็จ
"เทียบเคียง" กับแนวคิดใหม่ล่าสุดแล้ว)
ระดับของการบริหารที่ต้องใช้ในการผลักดันไปสู่แนวรุกใหม่จะมากมายมหาศาล''

มีข้อผิดพลาดร้ายแรงหลายอย่างเกี่ยวกับข้อเท็จจริงที่ข้อโต้แย้งนี้พยายามจะบอก
ข้อแรกถูกเปิดเผยเมื่อผู้เขียนเอกสารฮัลโลวีนเองได้ตั้งข้อสังเกตในภายหลังว่า
``บ่อยครั้ง {[}...{]} ที่แนวคิดงานวิจัยใหม่ ๆ  ถูกทำให้เป็นจริง
และมีให้ใช้ในลินุกซ์ก่อนที่จะมีหรือถูกรวมเข้าในแพล็ตฟอร์มอื่น''

ถ้าเราแทน ``ลินุกซ์'' ด้วย ``โอเพนซอร์ส''
เราจะเห็นว่าเรื่องนี้ไม่ใช่เรื่องใหม่อะไรเลย ตามประวัติแล้ว
ชุมชนโอเพนซอร์สไม่ได้ประดิษฐ์ Emacs หรือ World Wide Web
หรือตัวอินเทอร์เน็ตเองด้วยการไล่กวดไฟท้าย
หรือต้องมีการบริหารอย่างมากมายมหาศาลเลย และในปัจจุบัน
ก็มีงานนวัตกรรมมากมายที่ยังดำเนินต่อไปในโลกโอเพนซอร์ส
จนถึงกับทำให้ผู้ใช้เคยตัวกับการมีทางเลือก โครงการ GNOME
(เพื่อเป็นตัวอย่างของอีกหลายโครงการ) ก็ยังคงผลักดัน GUI
ใหม่ล่าสุดและเทคโนโลยีออบเจ็กต์อย่างหนัก
มากพอที่จะดึงดูดความสนใจจากสื่อมวลชนสาขาคอมพิวเตอร์ที่อยู่นอกชุมชนลินุกซ์
และยังมีตัวอย่างอื่นอีกเป็นกองทัพ ซึ่งการเข้าไปดู
\href{http://freshmeat.net/}{Freshmeat} สักวันหนึ่ง
ก็สามารถพิสูจน์ได้อย่างรวดเร็ว

แต่มีข้อผิดพลาดที่พื้นฐานกว่านั้นอีกเรื่องหนึ่ง คือการทึกทักกลาย ๆ  ว่า
{[}\emph{รูปแบบมหาวิหาร}{]} (หรือรูปแบบตลาดสด
หรือรูปแบบการบริหารชนิดอื่นใด) สามารถสร้างนวัตกรรมได้อย่างแน่นอน
นี่เป็นเรื่องไร้สาระ กลุ่มคนใด ๆ  ไม่สามารถมีแนวคิดที่พลิกโฉมได้
แม้กลุ่มอาสาสมัครอนาธิปัตย์ในตลาดสด
ก็มักไม่สามารถคิดค้นอะไรใหม่ได้อย่างแท้จริง
ไม่ต้องพูดถึงกลุ่มคณะกรรมการในบริษัท
ที่มีความเสี่ยงต่อการอยู่รอดโดยมีสถานะของบริษัทเป็นเดิมพันเลย แต่
{[}\emph{แนวคิดมาจากปัจเจกบุคคล}{]} ต่างหาก
สิ่งที่ดีที่สุดที่กลไกทางสังคมที่แวดล้อมเขาสามารถคาดหวังว่าจะทำได้
ก็คือการ {[}\emph{ตอบสนอง}{]} ต่อแนวคิดที่พลิกโฉมต่าง ๆ
โดยหล่อเลี้ยงและตอบแทนและทดสอบแนวคิดอย่างจริงจัง แทนที่จะขยี้ทิ้งเสีย

บางคนอาจจะมองว่านี่เป็นมุมมองเพ้อฝัน
ที่ย้อนกลับไปสู่เรื่องของบุคคลตัวอย่างที่เป็นผู้ประดิษฐ์โดยลำพังในแบบเก่า
ไม่ใช่อย่างนั้น ผมไม่ได้อ้างว่ากลุ่มคนจะไม่สามารถ {[}\emph{พัฒนา}{]}
แนวคิดพลิกโฉมได้หลังจากที่ได้เกิดแนวคิดขึ้นแล้ว อันที่จริง
เราได้เรียนรู้จากกระบวนการตรวจทานโดยนักพัฒนาอื่นมาแล้ว
ว่ากลุ่มพัฒนาลักษณะนั้นมีความสำคัญต่อการสร้างผลลัพธ์คุณภาพสูง
แต่ผมกำลังชี้ให้เห็นว่า การพัฒนาในแต่ละกลุ่มดังกล่าว จะเริ่มจาก
(และต้องถูกจุดประกายโดย) แนวคิดที่ดีในหัวของคนคนหนึ่ง
มหาวิหารและตลาดสดและโครงสร้างทางสังคมแบบอื่นสามารถจับประกายนั้น
แล้วปรับปรุงต่อได้ แต่จะไม่สามารถสร้างขึ้นเองได้ตามต้องการ

ดังนั้น ต้นตอของปัญหาของนวัตกรรม (ในซอฟต์แวร์ หรือในสาขาอื่น ๆ )
โดยเนื้อแท้ จึงอยู่ที่การทำอย่างไรไม่ให้นวัตกรรมต้องถูกทิ้งไป
แต่ที่อาจจะพื้นฐานกว่านั้น คือ
{[}\emph{ทำอย่างไรจึงจะสร้างกลุ่มคนที่สามารถมีแนวคิดดี ๆ
      ได้ตั้งแต่ต้น}{]}

การทึกทักว่าการพัฒนาในแบบมหาวิหารจะสามารถจัดการเคล็ดลับนี้ได้
แต่แนวกั้นต่อการเข้าร่วมที่ต่ำ
และความคล่องตัวของกระบวนการของตลาดสดจะทำไม่ได้ จึงเป็นเรื่องน่าขัน
ถ้าสิ่งที่ต้องการมีแค่คนคนเดียวที่มีความคิดที่ดีแล้วล่ะก็
สภาพแวดล้อมทางสังคมที่คนคนหนึ่งสามารถดึงดูดความร่วมมือของคนอื่นเป็นร้อยเป็นพันที่มีแนวคิดที่ดี
ก็เลี่ยงไม่ได้ที่จะสร้างนวัตกรรมแซงหน้ารูปแบบใด ๆ
ที่บุคคลต้องพยายามเสนอขายทางการเมืองให้กับผู้บริหารในทำเนียบ
ก่อนที่จะสามารถทำตามแนวคิดได้ โดยไม่เสี่ยงต่อการถูกไล่ออกจากงาน

และอันที่จริงแล้ว
ถ้าเราดูประวัติของนวัตกรรมซอฟต์แวร์ที่เกิดจากองค์กรที่ใช้รูปแบบมหาวิหารแล้ว
เราจะเห็นได้อย่างรวดเร็วว่าเกิดขึ้นน้อยมาก บริษัทใหญ่ ๆ
จะอาศัยแนวคิดใหม่ ๆ  จากงานวิจัยของมหาวิทยาลัย
(ทำให้ผู้เขียนเอกสารวันฮัลโลวีนไม่สบายใจนัก
เกี่ยวกับเครื่องไม้เครื่องมือของลินุกซ์
ที่เลือกหยิบใช้งานวิจัยเหล่านั้นได้รวดเร็วกว่า) หรือมิฉะนั้น
ก็ซื้อกิจการบริษัทเล็ก ๆ  ที่สร้างขึ้นจากมันสมองของผู้สร้างนวัตกรรมบางคน
ไม่มีกรณีใดที่นวัตกรรมจะเกิดจากวัฒนธรรมมหาวิหารโดยแท้จริงเลย อันที่จริง
นวัตกรรมหลายชิ้นที่นำเข้ามาด้วยวิธีดังกล่าว
กลับต้องขาดใจตายอย่างเงียบเชียบ ภายใต้ "ระดับการบริหารที่มากมายมหาศาล"
ที่ผู้เขียนเอกสารวันฮัลโลวีนสรรเสริญยิ่งนัก

อย่างไรก็ดี นั่นเป็นประเด็นเชิงลบ ผู้อ่านควรได้รับประเด็นเชิงบวกบ้าง
ผมขอแนะนำให้ทดลองดังนี้:

\begin{itemize}
      \item
            เลือกเกณฑ์สำหรับวัดจำนวนการคิดค้นที่คุณเชื่อว่าสามารถใช้ได้อย่างสม่ำเสมอ
            ถ้านิยามของคุณคือ ``ฉันรู้เมื่อได้เห็นก็แล้วกัน''
            นั่นก็ไม่ใช่ปัญหาสำหรับการทดลองนี้
      \item
            เลือกระบบปฏิบัติการซอร์สปิดตัวไหนก็ได้ที่แข่งกับลินุกซ์
            พร้อมทั้งแหล่งสำหรับตรวจสอบงานพัฒนาปัจจุบัน
      \item
            เฝ้าดูแหล่งดังกล่าวและ Freshmeat ทุกวันเป็นเวลาหนึ่งเดือน
            นับจำนวนการประกาศออกรุ่นที่ Freshmeat ที่คุณถือว่าเป็นงาน `คิดค้น'
            และใช้เกณฑ์เดียวกันของงาน `คิดค้น'
            นี้กับการประกาศของระบบปฏิบัติการอีกตัวนั้น แล้วนับดู
      \item
            สามสิบวันให้หลัง รวมคะแนนทั้งสองฝ่าย
\end{itemize}

ในวันที่ผมเขียนตรงนี้ Freshmeat มีประกาศออกรุ่นยี่สิบสองรายการ
ซึ่งมีสามรายการที่อาจเป็นสิ่งใหม่ล่าสุดในระดับหนึ่ง
นี่ยังถือเป็นวันที่เชื่องช้าสำหรับ Freshmeat
แต่ผมจะตกใจมากถ้ามีผู้อ่านท่านใดรายงานว่า
มีสิ่งที่อาจเป็นนวัตกรรมถึงสามรายการ {[}\emph{ต่อเดือน}{]}
ในแหล่งซอร์สปิดแหล่งไหน}

การทดสอบ ตรวจบั๊ก และปรับปรุงในแบบตลาดสดนั้น เป็นไปได้ แต่การ
{[}\emph{เริ่มต้น}{]} โครงการในแบบตลาดสดเป็นเรื่องที่ยากมาก
ไลนัสเองก็ไม่ได้ลองทำ ผมก็ไม่ได้ลองเหมือนกัน
ชุมชนนักพัฒนาที่จะบังเกิดขึ้น
จะต้องการอะไรที่ทำงานได้และทดสอบได้ไว้เล่นด้วย

เมื่อคุณเริ่มสร้างชุมชน สิ่งที่คุณต้องสามารถนำเสนอให้ได้ก็คือ
{[}\emph{ความเป็นไปได้}{]} โปรแกรมของคุณไม่จำเป็นต้องทำงานได้ดี
มันอาจจะหยาบ ๆ  มีบั๊ก ไม่สมบูรณ์ และขาดเอกสารอธิบายได้
แต่สิ่งที่จะพลาดไม่ได้ก็คือ (ก) ต้องเรียกใช้งานได้ (ข)
ทำให้ผู้ที่จะร่วมพัฒนาเชื่อได้
ว่ามันจะวิวัฒน์ไปเป็นสิ่งที่เนี้ยบได้ในอนาคตอันใกล้

ทั้งลินุกซ์และ fetchmail
ต่างออกสู่สาธารณะพร้อมการออกแบบที่แข็งแรงและดึงดูด
หลายคนที่คิดเกี่ยวกับรูปแบบตลาดสดตามที่ผมได้นำเสนอ
ต่างถือว่าสิ่งนี้สำคัญยิ่งยวด แล้วก็กระโดดจากจุดนั้นไปยังข้อสรุปทันที
ว่าความหยั่งรู้ในการออกแบบและความฉลาดของผู้นำโครงการ
เป็นสิ่งที่ขาดเสียไม่ได้

แต่ไลนัสได้การออกแบบของเขามาจากยูนิกซ์ ผมเองก็ได้การออกแบบของผมมาจาก
popclient ที่มีมาก่อน (แม้จะเปลี่ยนไปอย่างมากในภายหลัง
มากกว่าที่เกิดกับลินุกซ์หลายเท่า) ดังนั้น
ผู้นำหรือผู้ประสานงานของงานในแบบตลาดสด
ยังต้องมีพรสวรรค์ในการออกแบบอย่างเอกอุ
หรือเขาสามารถสร้างขึ้นได้จากการใช้พรสวรรค์การออกแบบของผู้อื่น?

ผมคิดว่าไม่ใช่เรื่องคอขาดบาดตายเลย
ที่ผู้ประสานงานจะต้องสามารถออกแบบซอฟต์แวร์ได้อย่างบรรเจิด
แต่เป็นเรื่องสำคัญมาก ที่ผู้ประสานงานต้องสามารถ
{[}\emph{ตระหนักรู้แนวคิดการออกแบบที่ดีจากผู้อื่น}{]}

ทั้งโครงการลินุกซ์และ fetchmail ต่างแสดงตัวอย่างของเรื่องนี้
ในขณะที่ไลนัสไม่ใช่นักออกแบบคิดค้นที่ดีเลิศ (ดังที่ได้กล่าวไปแล้ว)
แต่เขาได้แสดงความสามารถพิเศษในการตระหนักรู้การออกแบบที่ดี
และผนวกรวมเข้าในเคอร์เนลลินุกซ์ และผมก็ได้บรรยายไปแล้ว
ถึงการที่แนวคิดการออกแบบดีที่สุดเพียงชิ้นเดียวใน fetchmail
(การส่งเมลต่อไปยัง SMTP) มาจากคนอื่น

ผู้อ่านบทความนี้คนแรก ๆ  ได้ยกยอผม
โดยบอกว่าผมหมิ่นเหม่ที่จะประเมินคุณค่าของความเป็นผู้คิดค้นในโครงการตลาดสดต่ำเกินไป
เพราะผมมีความเป็นผู้คิดค้นอยู่ในตัว และก็เลยไม่พูดถึงมัน
ก็อาจจะมีส่วนจริง การออกแบบเป็นความเชี่ยวชาญที่แข็งที่สุดของผม
(ถ้าเทียบกับการเขียนโค้ดหรือการตรวจบั๊ก)

แต่ปัญหาของความฉลาดและความเป็นผู้คิดค้นในการออกแบบซอฟต์แวร์
ก็คือมันจะกลายเป็นนิสัย กล่าวคือ คุณจะเริ่มทำอะไรเจ๋ง ๆ
และซับซ้อนแบบเป็นไปโดยอัตโนมัติในจุดที่คุณควรทำให้มันเรียบง่ายและแน่นหนา
ผมเคยทำโครงการพังเพราะทำผิดแบบนี้มาแล้ว
แต่ผมพยายามหลีกเลี่ยงปัญหาดังกล่าวเมื่อทำ fetchmail

ดังนั้น ผมจึงเชื่อว่าโครงการ fetchmail
ประสบความสำเร็จเพราะผมงดนิสัยของผมที่จะพยายามฉลาด เรื่องนี้
อย่างน้อยก็สามารถค้านเรื่องที่ว่า
ความเป็นผู้คิดค้นออกแบบเป็นสิ่งจำเป็นสำหรับความสำเร็จของโครงการตลาดสด
และลองพิจารณาลินุกซ์ดูสิ สมมุติว่า ไลนัส ทอร์วัลด์
ได้พยายามดึงเอานวัตกรรมพื้นฐานของการออกแบบระบบปฏิบัติการออกไปในระหว่างการพัฒนา
มันจะเป็นไปได้ไหมที่เคอร์เนลที่ได้จะเสถียรและประสบความสำเร็จอย่างที่เรามี?

แน่นอนว่าความเชี่ยวชาญในขั้นพื้นฐานของการออกแบบและเขียนโค้ดเป็นสิ่งจำเป็น
แต่ผมก็คาดหวังว่าใครที่คิดจะตั้งโครงการแบบตลาดสดจริงจัง
ก็คงมีความสามารถเหนือระดับที่ต้องการอยู่แล้ว
ตลาดของชื่อเสียงภายในชุมชนโอเพนซอร์สได้ให้แรงกดดันอย่างละเอียดอ่อนต่อผู้คน
ที่จะไม่ตั้งโครงการพัฒนาที่ตัวเองไม่มีความสามารถจะดูแล เท่าที่ผ่านมา
แรงกดดันดังกล่าวก็ทำงานได้ดี

ยังมีความเชี่ยวชาญอีกชนิดหนึ่งที่ไม่ได้เกี่ยวข้องกับการพัฒนาซอฟต์แวร์
ที่ผมคิดว่าสำคัญต่อโครงการแบบตลาดสดพอ ๆ  กับความฉลาดในการออกแบบ
และอาจจะสำคัญกว่าเสียด้วยซ้ำ กล่าวคือ
ผู้ประสานงานหรือผู้นำโครงการแบบตลาดสด
จะต้องมีความเชี่ยวชาญในการติดต่อสื่อสารกับผู้คน

เรื่องนี้ควรจะชัดเจนอยู่แล้ว ในการสร้างชุมชนพัฒนา คุณต้องดึงดูดผู้คน
ทำให้เขาสนใจในสิ่งที่คุณทำ และทำให้เขายินดีกับปริมาณงานที่เขาทำ
งานทางเทคนิคจะมีอยู่ตลอดเพื่อทำงานนี้
แต่เป็นเพียงส่วนน้อยของเรื่องราวทั้งหมด
บุคลิกที่คุณแสดงออกก็มีความสำคัญเช่นกัน

ไม่ใช่เรื่องบังเอิญที่ไลนัสเป็นคนนิสัยดีที่ทำให้คนที่คล้ายกับเขาต้องการช่วยเขา
ไม่ใช่เรื่องบังเอิญที่ผมเป็นคนเปิดเผยที่มีชีวิตชีวาซึ่งชอบทำงานกับฝูงชน
และมีสัญชาตญาณของอารมณ์ขัน ถ้าจะให้รูปแบบตลาดสดทำงานได้
การมีความเชี่ยวชาญในการดึงดูดผู้คนจะช่วยได้มากทีเดียว