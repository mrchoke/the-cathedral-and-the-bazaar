\chapter{บทเรียนเพิ่มเติมจาก Fetchmail}

ก่อนที่เราจะกลับไปสู่ประเด็นเกี่ยวกับวิศวกรรมซอฟต์แวร์ทั่วไป
ยังมีบทเรียนให้ตรึกตรองอีกนิดหน่อยจากประสบการณ์ของ fetchmail โดยเฉพาะ
ผู้อ่านที่ไม่มีพื้นฐานทางเทคนิคสามารถข้ามหัวข้อนี้ไปได้

ไวยากรณ์ของแฟ้ม rc (แฟ้มควบคุม) มีคำแทรกที่จะใส่หรือไม่ก็ได้
ซึ่งตัวแจงจะไม่สนใจ ไวยากรณ์ที่คล้ายภาษาอังกฤษที่เกิดจากคำแทรกดังกล่าว
ทำให้อ่านง่ายกว่าการใช้คู่ คำหลัก-ค่า
ที่สั้นห้วนตามแบบฉบับที่คุณจะได้เมื่อตัดคำแทรกเหล่านั้นออกไป

แนวคิดดังกล่าว เริ่มจากการทดลองอะไรเล่น ๆ  ตอนดึก
หลังจากที่ผมสังเกตว่ารูปแบบการประกาศในแฟ้ม rc
ชักจะเริ่มคล้ายประโยคคำสั่งย่อย ๆ  (นี่เป็นเหตุผลที่ผมเปลี่ยนคำหลัก
``server'' ของ popclient ไปเป็น ``poll'')

ผมรู้สึกว่าการพยายามทำประโยคคำสั่งย่อย ๆ
ให้คล้ายภาษามนุษย์อาจทำให้มันใช้ง่ายขึ้น ทุกวันนี้
แม้ผมจะอยู่ฝ่ายสนับสนุนค่ายการออกแบบที่ ``ทำให้มันเป็นภาษา'' อย่างที่มี
Emacs และ HTML และโปรแกรมจัดการฐานข้อมูลหลายตัวเป็นตัวอย่าง แต่โดยปกติ
ผมก็ไม่ได้นิยมไวยากรณ์ที่ ``คล้ายภาษามนุษย์'' มากมายนัก

โดยธรรมเนียมแล้ว
โปรแกรมเมอร์มีแนวโน้มที่จะชอบไวยากรณ์ควบคุมที่เที่ยงตรงและกระชับ
และไม่มีส่วนเกินอยู่เลย
นี่เป็นมรดกทางวัฒนธรรมจากยุคที่ทรัพยากรการคำนวณมีราคาแพง
ซึ่งทำให้ขั้นตอนการแจงต้องประหยัดและง่ายที่สุดเท่าที่จะทำได้
ภาษามนุษย์ซึ่งมีส่วนเกินราว 50\% จึงไม่ใช่รูปแบบที่เหมาะสมในขณะนั้น

นี่ไม่ใช่เหตุผลที่โดยปกติผมพยายามหลีกเลี่ยงไวยากรณ์ที่คล้ายภาษามนุษย์
ผมยกเหตุผลดังกล่าวขึ้นมาในที่นี้เพียงเพื่อจะหักล้างเท่านั้น
ด้วยพลังคำนวณและหน่วยความจำที่ถูกลง
ความสั้นห้วนก็ไม่ควรจะเป็นคำตอบสุดท้ายอีกต่อไป ทุกวันนี้
สิ่งที่สำคัญกว่าสำหรับภาษา ก็คือความสะดวกต่อมนุษย์
ไม่ใช่ความง่ายสำหรับคอมพิวเตอร์

อย่างไรก็ดี ยังมีเหตุผลที่ดีที่พึงระวัง
ข้อแรกคือความซับซ้อนของขั้นตอนการแจง
คุณคงไม่ต้องการจะให้ความสำคัญกับมันจนกลายมาเป็นแหล่งบั๊กแหล่งใหญ่
หรือทำให้ผู้ใช้สับสน
อีกข้อหนึ่งคือการพยายามทำไวยากรณ์ของภาษาให้คล้ายภาษามนุษย์
มักจะติดข้อจำกัดว่า ``ภาษามนุษย์'' ที่ว่านั้น
จะถูกดัดจนผิดรูปอย่างร้ายแรง
จนถึงขั้นที่ความคล้ายคลึงกับภาษาธรรมชาติอย่างผิวเผินจะน่าปวดหัวพอ ๆ
กับไวยากรณ์แบบเก่า (คุณจะพบผลร้ายดังกล่าวได้ในสิ่งที่เรียกว่า ภาษา
``รุ่นที่สี่'' และภาษาสืบค้นฐานข้อมูลเชิงพาณิชย์ต่าง ๆ )

ไวยากรณ์ควบคุมของ fetchmail ดูจะหลีกเลี่ยงปัญหาดังกล่าว
เพราะกรอบของภาษาจะจำกัดอย่างมาก มันไม่มีอะไรคล้ายกับภาษาอเนกประสงค์เลย
สิ่งที่บรรยายเป็นเพียงสิ่งที่ไม่ซับซ้อน ดังนั้น
จึงมีแนวโน้มของความสับสนน้อยมากในการสลับไปมาระหว่างภาษามนุษย์ง่าย ๆ
กับภาษาที่ใช้ควบคุมจริง ผมคิดว่า
เราน่าจะได้บทเรียนที่กว้างขึ้นสำหรับเรื่องนี้:

\begin{fancyquotes}
  16. ตราบใดที่ภาษาของคุณไม่ได้ซับซ้อนระดับภาษาทูริงสมบูรณ์
  การเสริมแต่งไวยากรณ์ก็อาจช่วยคุณได้
\end{fancyquotes}

อีกบทเรียนหนึ่ง เป็นเรื่องเกี่ยวกับความนิรภัยโดยอาศัยความลึกลับ
(security by obscurity) ผู้ใช้ fetchmail
บางคนขอให้ผมแก้โปรแกรมให้เก็บรหัสผ่านในรูปที่เข้ารหัสลับในแฟ้ม rc
เพื่อไม่ให้ผู้บุกรุกอ่านรหัสผ่านได้ง่ายนัก

ผมไม่ทำตามคำขอนั้น เพราะมันไม่ได้เพิ่มการปกป้องใด ๆ  เลย
ใครก็ตามที่ได้รับสิทธิ์อ่านแฟ้ม rc ของคุณ จะสามารถเรียกใช้ fetchmail
ในฐานะตัวคุณได้อยู่แล้ว และถ้าเขากำลังล่าหารหัสผ่านของคุณ
เขาก็สามารถถอดส่วนถอดรหัสออกจากโค้ดของ fetchmail เพื่ออ่านเอาก็ได้

สิ่งที่จะได้จากการเข้ารหัสลับรหัสผ่านในแฟ้ม \texttt{.fetchmailrc}
ก็คือการให้มายาภาพของความนิรภัยต่อผู้ที่ไม่ได้คิดอย่างถี่ถ้วน
กฎทั่วไปของเรื่องนี้ก็คือ:

\begin{fancyquotes}
  17. ระบบนิรภัยจะมีความนิรภัยเท่ากับความลับที่มันเก็บเท่านั้น
  พึงระวังความนิรภัยหลอก ๆ
\end{fancyquotes}