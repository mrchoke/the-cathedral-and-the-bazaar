\chapter{เกี่ยวกับการบริหารจัดการและปัญหาที่ไม่เป็นปัญหา}

บทความ \emph{มหาวิหารและตลาดสด} ฉบับดั้งเดิมของปี 1997
จบลงด้วยภาพข้างต้น
คือภาพที่หมู่โปรแกรมเมอร์อนาธิปัตย์ที่ทำงานกันเป็นเครือข่ายอย่างมีความสุข
เอาชนะและท่วมทับโลกแห่งลำดับชั้นการบริหารของซอฟต์แวร์ปิดแบบเก่า

อย่างไรก็ดี ยังมีผู้กังขาอยู่มากพอควรที่ยังไม่เชื่อ และคำถามที่พวกเขาถาม
ก็สมควรแก่การพิจารณา ความเห็นแย้งต่อเหตุผลของตลาดสดส่วนใหญ่
มาลงเอยที่การอ้าง
ว่าฝ่ายสนับสนุนรูปแบบตลาดสดประเมินผลของการเพิ่มพูนผลิตภาพของระบบบริหารแบบเดิมต่ำเกินไป

ผู้จัดการโครงการพัฒนาซอฟต์แวร์แบบดั้งเดิมมักเห็นค้านว่า
ความไม่จริงจังของกลุ่มผู้ร่วมโครงการที่มารวมตัว เปลี่ยนแปลง
และสลายไปในโลกโอเพนซอร์สนั้น
จะหักล้างกับส่วนสำคัญของข้อดีที่เห็นได้ชัดในเรื่องจำนวน
ที่ชุมชนโอเพนซอร์สมีเหนือนักพัฒนาซอร์สปิดใด ๆ  พวกเขาจะตั้งข้อสังเกตว่า
ในการพัฒนาซอฟต์แวร์นั้น สิ่งที่สำคัญคือความพยายามที่ยั่งยืนจริง ๆ
ในระยะยาว และการที่ลูกค้าสามารถคาดหวังการลงทุนที่ต่อเนื่องในผลิตภัณฑ์ได้
ไม่ใช่แค่ว่ามีคนจำนวนมากแค่ไหนที่โยนกระดูกลงหม้อแล้วรอให้แกงอุ่น

มีประเด็นสำคัญอยู่ในข้อโต้แย้งนี้แน่นอน และอันที่จริง
ผมได้สร้างแนวคิดที่คาดหวังว่า คุณค่าของการบริการในอนาคต
จะเป็นกุญแจสำคัญของเศรษฐศาสตร์การผลิตซอฟต์แวร์ อยู่ในบทความ
\href{http://catb.org/~esr/writings/cathedral-bazaar/magic-cauldron/}{\emph{The
    Magic Cauldron}}

แต่ข้อโต้แย้งนี้ก็มีปัญหาใหญ่ซ่อนอยู่เช่นกัน คือการเข้าใจเอาว่า
การพัฒนาแบบโอเพนซอร์สจะไม่สามารถให้ความพยายามที่ยั่งยืนแบบนั้นได้
อันที่จริงแล้ว
ก็มีโครงการโอเพนซอร์สหลายโครงการที่ได้รักษาทิศทางที่สอดคล้อง
และชุมชนผู้ดูแลที่มีประสิทธิภาพมาเป็นเวลานานพอสมควร
โดยไม่ต้องอาศัยโครงสร้างของแรงจูงใจ
หรือสายงานการบังคับบัญชาชนิดที่การบริหารแบบเดิมเห็นว่าจำเป็นเลย
การพัฒนาบรรณาธิกรณ์ GNU Emacs เป็นตัวอย่างที่สุดขั้วและให้แง่คิดได้มาก
โครงการนี้ ได้ซึมซับเอาความพยายามของผู้ร่วมสมทบงานเป็นร้อย ๆ  คน ตลอดเวลา
15 ปี เข้ามารวมในวิสัยทัศน์ทางสถาปัตยกรรมที่เป็นอันหนึ่งอันเดียวกัน
แม้จะมีกิจกรรมเกิดขึ้นมากมาย และมีเพียงคนเดียว (คือตัวผู้เขียน Emacs
เอง) ที่ได้ทำงานอย่างตื่นตัวตลอดช่วงระยะเวลาดังกล่าว
ไม่มีโปรแกรมบรรณาธิกรณ์ซอร์สปิดตัวไหน ที่จะมีสถิติยาวนานเทียบเท่าได้

ตัวอย่างนี้ อาจเป็นเหตุผลให้เราตั้งคำถามกลับ
เกี่ยวกับข้อดีของการพัฒนาซอฟต์แวร์ที่บริหารในแบบเดิม
โดยไม่ได้เกี่ยวข้องกับส่วนที่เหลือของข้อโต้แย้งระหว่างรูปแบบมหาวิหารกับตลาดสดเลย
ถ้ามันเป็นไปได้สำหรับ GNU Emacs
ที่จะแสดงวิสัยทัศน์ทางสถาปัตยกรรมที่คงเส้นคงวาตลอดเวลา 15 ปี
หรือสำหรับระบบปฏิบัติการอย่างลินุกซ์ ที่จะทำอย่างเดียวกันได้ตลอดเวลา 8
ปี ท่ามกลางการเปลี่ยนแปลงอย่างรวดเร็วของเทคโนโลยีฮาร์ดแวร์และแพล็ตฟอร์ม
และถ้ามีโครงการโอเพนซอร์สที่มีการออกแบบสถาปัตยกรรมเป็นอย่างดีจำนวนมาก
ที่มีช่วงเวลาการทำงานเกิน 5 ปี (ซึ่งก็มีจริง ๆ ) แล้วล่ะก็
เราก็ต้องตั้งข้อสงสัยแล้วล่ะ ว่าค่าโสหุ้ยหนัก ๆ
ของการพัฒนาด้วยการบริหารแบบเก่า จะให้อะไรแก่เราเป็นการตอบแทน (ถ้ามี)

ไม่ว่ามันจะเป็นอะไร แต่ไม่ใช่การดำเนินการที่รับประกันได้ในเรื่องกำหนดการ
การใช้งบประมาณ หรือการทำได้ครบตามข้อกำหนดแน่ ๆ  หายากมากที่จะมีโครงการที่
`มีการจัดการ' ที่สามารถบรรลุเป้าหมายใดเป้าหมายหนึ่งได้
โดยไม่ต้องพูดถึงการบรรลุครบทั้งสามเป้าหมายเลย
และดูจะไม่ใช่ความสามารถในการปรับตัวตามการเปลี่ยนแปลงของเทคโนโลยี
และบริบททางเศรษฐกิจในช่วงชีวิตของโครงการเช่นกัน
ชุมชนโอเพนซอร์สได้พิสูจน์ให้เห็นถึงประสิทธิผลที่มากกว่า {[}\emph{มาก}{]}
ในแง่ดังกล่าว (เช่น ดังที่ใครก็สามารถตรวจสอบได้
โดยเปรียบเทียบประวัติศาสตร์ 30 ปีของอินเทอร์เน็ต กับครึ่งชีวิตที่สั้น ๆ
ของเทคโนโลยีเครือข่ายที่สงวนสิทธิ์ หรือเปรียบเทียบค่าโสหุ้ยของการย้ายจาก
16 บิต ไป 32 บิต ในไมโครซอฟท์วินโดวส์
กับการเปลี่ยนรุ่นของลินุกซ์ที่แทบไม่ต้องใช้ความพยายามเลยในช่วงเดียวกัน
ไม่ใช่แค่ตามสายการพัฒนาของอินเทลเท่านั้น
แต่รวมถึงการย้ายไปยังฮาร์ดแวร์ชนิดอื่นกว่า 12 ชนิด รวมถึงชิปแอลฟา 64
บิตด้วย)

สิ่งหนึ่งที่หลายคนคิดว่าการพัฒนาแบบเก่าจะให้ตอบแทนได้
คือใครบางคนที่จะให้สัญญาทางกฎหมาย
และค่าชดเชยสำหรับการฟื้นตัวในกรณีที่โครงการมีปัญหา
แต่เรื่องดังกล่าวเป็นเพียงมายาภาพ สัญญาอนุญาตซอฟต์แวร์เกือบทั้งหมด
ได้เขียนถึงการสงวนการรับผิดชอบแม้กระทั่งการรับประกันคุณค่าเชิงการค้า
ไม่ต้องพูดถึงเรื่องการทำงานเลย
และกรณีของการฟื้นตัวหลังซอฟต์แวร์ไม่ทำงานได้สำเร็จก็หายากเต็มที
หรือแม้จะมีเป็นเรื่องปกติ การรู้สึกสบายใจที่มีใครให้ฟ้องร้องได้
ก็เป็นการผิดประเด็น คุณไม่อยากขึ้นโรงขึ้นศาลหรอก
คุณต้องการซอฟต์แวร์ที่ทำงานได้ต่างหาก

\noindent ฉะนั้นแล้ว อะไรคือสิ่งที่จะได้จากการจ่ายค่าโสหุ้ยของการบริหารดังกล่าว?

เพื่อที่จะเข้าใจเรื่องดังกล่าว เราต้องเข้าใจเสียก่อน
ว่าผู้บริหารโครงการพัฒนาซอฟต์แวร์คิดว่าเขากำลังทำอะไรอยู่
หญิงสาวคนหนึ่งที่ผมรู้จัก ซึ่งดูจะถนัดเรื่องนี้ บอกว่า
การบริหารโครงการซอฟต์แวร์ประกอบด้วยหน้าที่ห้าอย่าง:

\begin{itemize}
  \item \emph{กำหนดเป้าหมาย} และทำให้ทุกคนเดินหน้าไปในทิศทางเดียวกัน
  \item \emph{ตรวจสอบ} และทำให้แน่ใจว่าไม่มีการข้ามรายละเอียดที่สำคัญ
  \item \emph{กระตุ้น} ผู้คนให้ทำงานส่วนที่น่าเบื่อ แต่จำเป็น
  \item \emph{แบ่งงาน} ให้กับบุคคลต่าง ๆ  เพื่อผลการทำงานที่ดี
  \item \emph{จัดสรรทรัพยากร} ที่จำเป็นสำหรับการดำเนินโครงการ
\end{itemize}

เป็นเป้าหมายที่คุ้มค่า ทุกข้อเลย แต่ในรูปแบบโอเพนซอร์ส
และในสภาพแวดล้อมทางสังคมที่เกี่ยวข้อง เรื่องต่าง ๆ
ดังกล่าวกลับเริ่มจะไม่มีผลอย่างน่าประหลาด
เราจะพิจารณาแต่ละข้อในลำดับย้อนกลับ

เพื่อนของผมรายงานว่า {[}\emph{การจัดสรรทรัพยากร}{]}
โดยทั่วไปเป็นการปกป้อง เมื่อคุณมีคน มีเครื่อง และมีพื้นที่สำนักงานแล้ว
คุณต้องปกป้องสิ่งเหล่านั้นจากผู้จัดการโครงการอื่น ๆ
ที่จะแข่งขันยื้อแย่งทรัพยากรเดียวกัน
และจากผู้บริหารระดับบนที่พยายามจะจัดสรรทรัพยากรที่มีจำกัด
เพื่อใช้ให้เกิดประโยชน์สูงสุด

แต่นักพัฒนาโอเพนซอร์สล้วนแต่เป็นอาสาสมัคร และได้คัดเลือกตัวเอง
ทั้งในเรื่องความสนใจและความสามารถในการร่วมงานกับโครงการที่ตนทำงานอยู่แล้ว
(และโดยทั่วไป เรื่องนี้ก็ยังเป็นจริง
แม้ในกรณีที่ถูกจ้างด้วยเงินเดือนให้แฮ็กโอเพนซอร์ส)
แนวความคิดของตัวอาสาสมัครเองมีแนวโน้มจะดูแลด้าน `รุกล้ำ'
ของการจัดสรรทรัพยากรโดยอัตโนมัติอยู่แล้ว
ผู้คนจะนำทรัพยากรของตนมาใช้ในงานเอง
และผู้จัดการโครงการก็มีความจำเป็นน้อยมาก หรือไม่มีความจำเป็นเลย
ที่จะต้อง `เล่นบทปกป้อง' ในความหมายปกติ

อย่างไรก็ดี ในโลกของพีซีราคาถูก และอินเทอร์เน็ตความเร็วสูง
เราพบอยู่เนือง ๆ  ว่าทรัพยากรที่มีจำกัดเพียงอย่างเดียว
ก็คือความสนใจของคนที่เชี่ยวชาญ โครงการโอเพนซอร์สเมื่อจะล่มนั้น
ไม่ได้ล่มเพราะขาดเครื่องหรือเครือข่ายหรือพื้นที่สำนักงานเลย
แต่จะตายเมื่อนักพัฒนาเองขาดความสนใจในโครงการอีกต่อไปเท่านั้น

เมื่อเป็นดังนั้น เรื่องที่สำคัญเป็นสองเท่า ก็คือแฮ็กเกอร์โอเพนซอร์สนั้น
{[}\emph{แบ่งงานตัวเอง}{]} เพื่อผลิตภาพสูงสุดด้วยการพิจารณาตัวเอง
อีกทั้งสภาพแวดล้อมทางสังคมก็จะเลือกคนตามความสามารถอย่างไร้ความปรานี
เพื่อนผมซึ่งคุ้นเคยกับทั้งโลกโอเพนซอร์สและโครงการปิดขนาดใหญ่
เชื่อว่าที่โอเพนซอร์สประสบความสำเร็จ
ส่วนหนึ่งเป็นเพราะวัฒนธรรมของมันยอมรับเฉพาะผู้มีพรสวรรค์ 5\%
แรกของประชากรโปรแกรมเมอร์เท่านั้น และเธอใช้เวลาเกือบทั้งหมดของเธอ
ในการบริหารการใช้งานประชากรอีก 95\% ที่เหลือ
และเธอจึงได้มีโอกาสสังเกตโดยตรง ถึงความแตกต่างของผลิตภาพถึงร้อยเท่า
ระหว่างโปรแกรมเมอร์ที่เก่งสุด ๆ  กับโปรแกรมเมอร์ที่แค่มีความสามารถธรรมดา

ขนาดของความแตกต่างดังกล่าว ได้ทำให้เกิดคำถามเสมอ ๆ  ว่า
โครงการแต่ละโครงการ รวมทั้งภาพรวมของวงการทั้งหมด จะดีขึ้นไหม
ถ้าไม่มีคนกว่า 50\% ที่ด้อยความสามารถเหล่านั้น?
ผู้บริหารโครงการที่ชอบใคร่ครวญ จะเข้าใจมาเป็นเวลานานแล้ว
ว่าถ้าหน้าที่เดียวของการบริหารโครงการซอฟต์แวร์แบบเดิม
คือการเปลี่ยนคนที่เก่งน้อยที่สุด จากการขาดทุนสุทธิ
ให้ไปเป็นกำไรแล้วล่ะก็ จะเป็นเรื่องที่ไม่คุ้มค่าเอาเสียเลย

ความสำเร็จของชุมชนโอเพนซอร์สได้เพิ่มความแหลมคมของคำถามนี้ขึ้นไปอีก
โดยให้หลักฐานที่ชัดเจน
ว่าการใช้อาสาสมัครจากอินเทอร์เน็ตที่ได้กลั่นกรองตัวเองมาก่อนแล้ว
จะเสียค่าใช้จ่ายน้อยกว่า
และมีประสิทธิภาพกว่าการบริหารตึกที่เต็มไปด้วยคนที่อาจจะอยากทำอย่างอื่นมากกว่า

ซึ่งนำเรามาสู่คำถามเรื่อง {[}\emph{การกระตุ้น}{]} อย่างเหมาะเจาะ
วิธีที่เทียบเคียงและได้ยินบ่อย ๆ  ในการบรรยายประเด็นของเพื่อนผม
คือว่าการบริหารงานพัฒนาแบบเดิมนั้น
เป็นการชดเชยที่จำเป็นสำหรับโปรแกรมเมอร์ที่ขาดแรงจูงใจ
ซึ่งจะไม่ทำงานให้ดีถ้าไม่มีการกระตุ้น

คำตอบนี้ มักจะมาพร้อมกับการกล่าวอ้าง ว่าชุมชนโอเพนซอร์สจะพึ่งพาได้
ก็แต่สำหรับงานที่ `เจ๋ง' หรือเย้ายวนทางเทคนิคเท่านั้น
ส่วนที่เหลือจะถูกทิ้งร้าง (หรือทำอย่างลวก ๆ )
ถ้าไม่ถูกปั่นโดยผู้ใช้แรงงานในคอกที่มีเงินกระตุ้น
มีผู้บริหารโครงการโบยแส้ให้ทำงาน
ผมได้ให้เหตุผลทางจิตวิทยาและสังคมสำหรับการตั้งข้อสงสัยในข้ออ้างนี้ใน
\href{http://catb.org/~esr/writings/cathedral-bazaar/homesteading/}{\emph{Homesteading
    the Noosphere}} แต่อย่างไรก็ดี เพื่อจุดประสงค์ในขณะนี้
ผมคิดว่าการชี้ถึงนัยของการยอมรับว่าเรื่องนี้เป็นเรื่องจริง จะน่าสนใจกว่า

ถ้ารูปแบบการพัฒนาซอฟต์แวร์ในแบบเดิมซึ่งปิดซอร์สและมีการจัดการอย่างเข้มข้น
จะมีเหตุผลสนับสนุนเพียงเพราะเรื่องปัญหาที่นำไปสู่ความเบื่อหน่ายแล้วล่ะก็
มันก็จะมีผลอยู่ในแต่ละส่วนของโปรแกรมในระหว่างที่ไม่มีใครเห็นว่าปัญหานั้นน่าสนใจ
และไม่มีใครอีกที่จะกล้ำกรายเข้าไปใกล้เท่านั้น
เพราะเมื่อมีการแข่งขันของโอเพนซอร์สเกี่ยวกับซอฟต์แวร์ส่วนที่ `น่าเบื่อ'
ผู้ใช้ก็จะรู้เอง
ว่าปัญหาจะถูกแก้ในที่สุดโดยใครบางคนที่เลือกปัญหานั้นเพราะความน่าสนใจของปัญหาเอง
ซึ่งสำหรับซอฟต์แวร์ในฐานะงานสร้างสรรค์แล้ว
เป็นสิ่งกระตุ้นที่ได้ผลกว่าเงินเพียงอย่างเดียว

ดังนั้น
การมีโครงสร้างการบริหารแบบเดิมเพียงอย่างเดียวเพื่อสร้างแรงกระตุ้น
จึงอาจเป็นเทคนิคที่ดี แต่เป็นกลยุทธ์ที่แย่
เพราะอาจจะได้ประโยชน์ในระยะสั้นก็จริง แต่ในระยะยาวแล้ว
จะสูญเสียอย่างแน่นอน

เท่าที่ผ่านมา การบริหารงานพัฒนาแบบเดิม
ดูจะไม่ใช่ทางเลือกที่ดีเมื่อเทียบกับโอเพนซอร์สในสองประเด็น
(การจัดสรรทรัพยากร และการแบ่งงาน)
และดูเหมือนจะใช้ได้แค่ชั่วคราวในประเด็นที่สาม (การกระตุ้น)
และผู้บริหารโครงการแบบเก่าผู้ถูกคุกคามอย่างน่าสงสาร
ก็จะไม่ได้คะแนนช่วยจากประเด็น {[}\emph{การตรวจสอบ}{]} เลย
ข้อโต้แย้งที่แข็งที่สุดที่ชุมชนโอเพนซอร์สมี
ก็คือว่าการตรวจทานโดยนักพัฒนาอื่นแบบกระจายกำลัง จะชนะวิธีเดิม ๆ
ทั้งหมดในการตรวจสอบให้แน่ใจว่าไม่มีรายละเอียดต่าง ๆ  หลุดรอดไป

เราสามารถเก็บประเด็น {[}\emph{การกำหนดเป้าหมาย}{]}
ไว้เป็นเหตุผลสำหรับการยอมเสียค่าโสหุ้ยให้กับการบริหารโครงการซอฟต์แวร์แบบเดิมได้ไหม?
ก็อาจจะได้ แต่การจะยอมรับ เราก็ต้องการเหตุผลที่ดีที่จะเชื่อ
ว่าคณะกรรมการบริหารและแผนงานของบริษัท
จะประสบความสำเร็จในการกำหนดเป้าหมายที่คุ้มค่าและเห็นร่วมกันอย่างกว้างขวาง
มากกว่าผู้นำโครงการและสมาชิกอาวุโสซึ่งทำหน้าที่คล้ายกันนี้ในโลกโอเพนซอร์ส

เรื่องนี้ค่อนข้างจะเชื่อได้ยาก
ไม่ใช่เพราะข้อมูลสนับสนุนของฝ่ายโอเพนซอร์ส (ความยืนยาวของ Emacs
หรือความสามารถของ ไลนัส ทอร์วัลด์
ในการขับเคลื่อนหมู่นักพัฒนาด้วยการพูดเกี่ยวกับ ``การครองโลก'')
ที่ทำให้เชื่อได้ยาก
แต่เป็นเพราะความน่ากลัวที่ได้แสดงให้เห็นของกลไกแบบเดิม
ในการกำหนดเป้าหมายของโครงการซอฟต์แวร์มากกว่า

ทฤษฎีชาวบ้านที่รู้จักกันดีที่สุดทฤษฎีหนึ่งของวิศวกรรมซอฟต์แวร์
ก็คือร้อยละ 60 ถึง 70 ของโครงการซอฟต์แวร์แบบเดิม จะไม่เคยเสร็จ
หรือไม่ก็ถูกผู้ใช้กลุ่มเป้าหมายปฏิเสธ
ถ้าอัตราส่วนดังกล่าวใกล้เคียงกับความเป็นจริง
(ซึ่งผมก็ไม่เคยเห็นผู้บริหารที่มีประสบการณ์คนไหนกล้าเถียง) ก็หมายความว่า
มีโครงการส่วนใหญ่ที่กำลังมุ่งสู่เป้าหมายที่ (ก)
ไม่สามารถบรรลุได้ในความเป็นจริง หรือ (ข) ผิดพลาด

เรื่องนี้เป็นเหตุผลมากกว่าเรื่องอื่น ๆ
ที่ทำให้ในโลกวิศวกรรมซอฟต์แวร์ทุกวันนี้ วลี ``คณะบริหาร''
มักจะทำให้ผู้ที่ได้ยินต้องเสียวสันหลังวาบ แม้ว่า (หรือบางที โดยเฉพาะถ้า)
ผู้ที่ได้ยินนั้นเป็นผู้บริหารเช่นกัน
วันที่เรื่องนี้เป็นที่รู้กันเฉพาะในหมู่โปรแกรมเมอร์ ได้ผ่านไปนานแล้ว
การ์ตูนดิลเบิร์ตทุกวันนี้ ไปไม่ค่อยพ้นโต๊ะของ {[}\emph{ผู้บริหาร}{]}
หรอก

ดังนั้น คำตอบของเราสำหรับนักบริหารโครงการพัฒนาซอฟต์แวร์แบบเดิม
จึงง่ายดาย กล่าวคือ
ถ้าชุมชนโอเพนซอร์สประเมินค่าของการบริหารแบบเดิมต่ำเกินไปแล้วล่ะก็
{[}\emph{ทำไมพวกคุณจำนวนมากถึงได้ดูแคลนกระบวนการของคุณเองเล่า?}{]}

อีกครั้งที่ตัวอย่างของชุมชนโอเพนซอร์สได้เพิ่มความแหลมคมให้กับคำถามนี้อย่างชัดเจน
เพราะเรา {[}\emph{สนุก}{]} กับสิ่งที่เราทำนั่นเอง
งานสร้างสรรค์ของเราประสบผลสำเร็จทั้งทางเทคนิค ส่วนแบ่งตลาด และความยอมรับ
ในอัตราที่น่าอัศจรรย์ เราได้พิสูจน์แล้ว
ไม่เพียงแค่ว่าเราสามารถพัฒนาซอฟต์แวร์ที่ดีกว่าได้ แต่ยังพิสูจน์ด้วยว่า
{[}\emph{ความรื่นเริงคือทรัพย์สิน}{]}

สองปีครึ่งหลังจากเขียนบทความนี้รุ่นแรก
แนวคิดระดับรากฐานที่สุดที่ผมสามารถให้ได้เพื่อสรุปเรื่องทั้งหมด
ไม่ใช่วิสัยทัศน์ของโลกซอฟต์แวร์ที่โอเพนซอร์สเป็นใหญ่
ซึ่งดูเป็นไปได้สำหรับคนปกติทั่วไปในทุกวันนี้

แต่ผมต้องการชี้แนะในสิ่งที่อาจเป็นบทเรียนที่กว้างขึ้นเกี่ยวกับซอฟต์แวร์
(และอาจจะเกี่ยวกับงานสร้างสรรค์และงานวิชาชีพทุกชนิด)
มนุษย์มักจะมีความยินดีกับงานเมื่อมันมีความท้าทายที่เหมาะสม
ไม่ง่ายจนน่าเบื่อ และไม่ยากจนไม่สามารถบรรลุได้ โปรแกรมเมอร์ที่มีความสุข
ก็คือคนที่ไม่ถูกใช้งานต่ำเกินไป หรือต้องแบกรับเป้าหมายที่กำหนดไว้ชุ่ย ๆ
โดยมีแรงเสียดทานต่อการทำงานอย่างเคร่งเครียด
{[}\emph{ความรื่นเริงจะให้ประสิทธิภาพ}{]}

ถ้าคุณรู้สึกถึงความเกลียดและความกลัวเมื่อนึกถึงกระบวนการทำงานของคุณ
(แม้จะในแบบประชดประชันตามแบบการ์ตูนดิลเบิร์ตก็ตาม)
ก็ควรถือว่านั่นเป็นเครื่องหมายบ่งชี้ว่ากระบวนการได้ล้มเหลวเสียแล้ว
ความรื่นเริง อารมณ์ขัน และความขี้เล่น เป็นทรัพย์สินอย่างแท้จริง
มันไม่ใช่การทำให้ดูเพราะพริ้งเมื่อผมเขียนถึง
"หมู่โปรแกรมเมอร์ผู้มีความสุข"
และก็ไม่ใช่เรื่องขบขันเลยที่สัตว์นำโชคของลินุกซ์เป็นนกเพนกวินแรกรุ่นอ้วนจ้ำม่ำ

กลายเป็นว่า ผลที่สำคัญที่สุดอย่างหนึ่งของความสำเร็จของโอเพนซอร์ส
ก็คือการสอนเรา
ว่าการเล่นเป็นวิธีที่มีประสิทธิภาพทางเศรษฐกิจที่สุดสำหรับงานสร้างสรรค์
