\chapter{ความสำคัญของการมีผู้ใช้}

และแล้วผมก็รับช่วง popclient ต่อมา และที่สำคัญไม่แพ้กันคือ
ผมรับช่วงฐานผู้ใช้ของ popclient มาด้วย การมีผู้ใช้เป็นสิ่งที่ยอดมาก
ไม่ใช่แค่เพราะพวกเขาแสดงให้เห็น ว่าคุณกำลังสนองความต้องการที่มีอยู่จริง
ว่าคุณกำลังทำสิ่งที่ควร แต่ถ้าบ่มเพาะดี  ๆ
เขาก็อาจมาเป็นผู้ร่วมพัฒนากับคุณได้

จุดแข็งอีกอย่างหนึ่งในขนบของยูนิกซ์ ซึ่งลินุกซ์ผลักดันไปถึงจุดสุดยอด
คือผู้ใช้หลายคนเป็นแฮ็กเกอร์ด้วย และเพราะมีซอร์สโค้ดให้ดู
พวกเขาจึงเป็นแฮ็กเกอร์ที่ {[}\emph{ทรงประสิทธิภาพ}{]}
ซึ่งมีประโยชน์อย่างมหาศาลในการย่นเวลาจัดการบั๊ก
เพียงแค่คุณกระตุ้นพวกเขาสักเล็กน้อย เหล่าผู้ใช้จะช่วยกันหาสาเหตุของปัญหา
และแนะนำวิธีแก้ไข แถมช่วยพัฒนาโค้ดได้เร็วกว่าที่คุณทำเองคนเดียวเสียอีก

\begin{fancyquotes}
  6. การปฏิบัติต่อผู้ใช้เยี่ยงผู้ร่วมงาน เป็นหนทางที่สะดวกที่สุด
  ที่จะนำไปสู่การพัฒนาโค้ดอย่างรวดเร็ว และการแก้บั๊กอย่างได้ผล
\end{fancyquotes}

เรามักประเมินผลของวิธีการแบบนี้ต่ำไป ความจริงแล้ว
พวกเราส่วนใหญ่ในโลกโอเพนซอร์สประเมินพลาดไปอย่างแรง
ว่ามันสามารถรองรับจำนวนผู้ใช้ที่เพิ่มขึ้น
และเอาชนะความซับซ้อนของระบบได้ดีเพียงใด จนกระทั่ง ไลนัส ทอร์วัลด์
ได้แสดงให้เราเห็น ว่ามันไม่ได้เป็นอย่างที่เราคิด

ความจริงแล้ว ผมคิดว่าผลงานที่ชาญฉลาดที่สุด
และส่งผลต่อเนื่องมากที่สุดของไลนัส ไม่ใช่การสร้างเคอร์เนลลินุกซ์
แต่เป็นการสร้างตัวแบบการพัฒนาลินุกซ์ต่างหาก ครั้งหนึ่ง
ผมเคยแสดงความเห็นนี้ต่อหน้าเขา เขาก็ยิ้ม
และกล่าวสิ่งที่เขาพูดอยู่เสมอด้วยเสียงเบา  ๆ   ว่า ``ผมเป็นแค่คนขี้เกียจสุด  ๆ
คนหนึ่ง ที่อยากได้ชื่อจากผลงานของคนอื่น'' ขี้เกียจอย่างหมาจิ้งจอก
หรืออย่างที่ โรเบิร์ต ไฮน์ไลน์
นักเขียนชื่อดังได้บรรยายตัวละครของเขาตัวหนึ่งไว้ว่า
``ขี้เกียจเกินกว่าจะล้มเหลว''

ถ้าเรามองย้อนกลับไป วิธีการและความสำเร็จของลินุกซ์ เคยเกิดขึ้นมาก่อนแล้ว
ในการพัฒนาไลบรารี Lisp พร้อมทั้งคลังโค้ด Lisp ของโครงการ GNU Emacs
ซึ่งตรงข้ามกับการพัฒนาแบบมหาวิหารอย่างส่วนหลักที่เขียนด้วยภาษาซีของ
Emacs และโปรแกรมส่วนมากของ GNU วิวัฒนาการของโค้ดส่วน Lisp
นั้นเป็นไปอย่างลื่นไหล และถูกผลักดันโดยผู้ใช้อย่างมาก
แนวคิดและตัวต้นแบบมักจะถูกแก้ไขและเขียนใหม่ 3-4 ครั้ง
ก่อนจะได้รูปแบบสุดท้ายที่เสถียรพอ และการร่วมมือกันอย่างหลวม  ๆ
ที่ทำผ่านอินเทอร์เน็ต ก็เกิดขึ้นอยู่เสมอ เช่นเดียวกับลินุกซ์

อันที่จริง ก่อนผมจะทำ fetchmail
ผลงานที่ผมคิดว่าประสบความสำเร็จที่สุดของผมนั้น คงเป็นโหมด VC (Version
Control) ของ Emacs โดยเป็นการทำงานแบบลินุกซ์
คือร่วมมือกันผ่านเมลกับคนอื่นอีก 3 คน ซึ่งจนถึงทุกวันนี้
มีเพียงคนเดียวในกลุ่มนั้น (คือ ริชาร์ด สตอลล์แมน ผู้สร้าง Emacs
และผู้ก่อตั้ง \href{http://www.fsf.org}{มูลนิธิซอฟต์แวร์เสรี})
ที่ผมเคยพบหน้า โหมด VC เป็นส่วนติดต่อ (front end) กับ SCCS, RCS
และต่อมาถึง CVS ของ Emacs
ที่ช่วยให้การดำเนินการกับระบบควบคุมเวอร์ชันเป็นไปได้ใน ``สัมผัสเดียว''
เป็นการพัฒนามาจากโหมด sccs.el เล็ก  ๆ   ง่าย  ๆ   ที่ใครบางคนเขียนไว้ การพัฒนา
VC ประสบความสำเร็จเพราะ Emacs Lisp ได้ผ่านขั้นตอนต่าง  ๆ
ของวัฏจักรซอฟต์แวร์ (ออก/ทดสอบ/พัฒนา) อย่างรวดเร็ว ไม่เหมือนตัว Emacs
เอง

เรื่องทำนองนี้ไม่ได้เกิดกับ Emacs เท่านั้น
ยังมีซอฟต์แวร์อื่นที่มีโครงสร้างสองชั้น พร้อมชุมชนผู้ใช้สองระดับ
ซึ่งประกอบด้วยแกนที่พัฒนาแบบมหาวิหาร และชุดเครื่องมือที่พัฒนาแบบตลาดสด
ตัวอย่างหนึ่งคือ MATLAB
เครื่องมือเชิงพาณิชย์สำหรับวิเคราะห์และพล็อตภาพข้อมูล ผู้ใช้ MATLAB
และผลิตภัณฑ์อื่นทำนองนี้ต่างรายงานเหมือน  ๆ   กัน ว่าความเคลื่อนไหว
ความตื่นตัว และนวัตกรรม มักเกิดในส่วนที่เปิด
ซึ่งชุมชนที่ใหญ่โตและหลากหลายสามารถเข้าไปเปลี่ยนแปลงอะไรเองได้