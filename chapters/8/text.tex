\chapter{Fetchmail เติบโต}

แล้วผมก็อยู่กับการออกแบบที่เนี้ยบและมีนวัตกรรม
อยู่กับโค้ดที่ผมรู้ว่าทำงานได้ดี เพราะผมใช้อยู่ทุกวัน
และอยู่กับผู้ทดสอบเบต้าที่เพิ่มขยายขึ้นเรื่อย ๆ  ผมเริ่มรู้สึกทีละนิด
ว่าผมไม่ได้ผูกพันกับการแฮ็กเล็ก ๆ  น้อย ๆ
ของตัวเองที่อาจจะบังเอิญมีประโยชน์กับคนอื่นบางคนอีกต่อไป
แต่ผมกำลังเขียนโปรแกรมที่แฮ็กเกอร์ที่ใช้ยูนิกซ์และอ่านเมลผ่าน SLIP/PPP
ทุกคนต้องมี

ด้วยความสามารถส่งเมลต่อไปยัง SMTP ทำให้ fetchmail
นำหน้าคู่แข่งไปไกลจนถึงขั้นสามารถเป็น ``killer''
หรือโปรแกรมอมตะที่เติมช่องว่างได้อย่างเฉียบขาด
จนทำให้ตัวเลือกอื่นไม่ใช่แค่ถูกทิ้งไป แต่แทบจะถูกลืมไปเลย

ผมคิดว่าคุณไม่อาจจะตั้งเป้าหรือวางแผนเพื่อให้โปรแกรมมาถึงจุดนี้ได้เลย
คุณต้องเป็นไปเอง
ด้วยแนวคิดการออกแบบที่ทรงพลังพอที่ผลลัพธ์ที่ออกมาจะกลายเป็นสิ่งที่เลี่ยงไม่ได้
เป็นธรรมชาติ หรือแม้แต่เหมือนถูกลิขิตไว้
วิธีเดียวที่จะได้สุดยอดแนวคิดอย่างนั้นมา คือต้องมีความคิดจำนวนมาก
หรือไม่ก็มีวิจารณญาณทางวิศวกรรมที่นำความคิดของผู้อื่นมาใช้
โดยที่เจ้าของไม่นึกว่าจะนำมาใช้ได้ขนาดนี้

แอนดี้ ทาเนนบอม มีความคิดเริ่มแรกคือสร้างยูนิกซ์ง่าย ๆ  สำหรับ IBM PC
โดยเฉพาะ เพื่อใช้เป็นตัวอย่างในการสอนหนังสือ (เขาเรียกมันว่ามินิกซ์)
ไลนัส ทอร์วัลด์
ผลักดันแนวคิดของมินิกซ์ต่อไปจนไกลเกินกว่าที่แอนดี้จะเคยนึกถึง
และมันก็กลายเป็นสิ่งมหัศจรรย์ ในทำนองเดียวกัน (แต่ขนาดเล็กกว่า)
ผมนำแนวคิดบางอย่างของ คาร์ล แฮร์ริส และ Harry Hochheiser มาใช้
แล้วผลักดันต่ออย่างจริงจัง ในบรรดาพวกเรา ไม่มีใครสักคนที่เป็น
`ผู้คิดค้น' ในแบบที่ผู้คนนึกฝันว่าเป็นอัจฉริยะเลย แต่จริง ๆ
แล้วผลงานทางวิทยาศาสตร์ วิศวกรรม และการพัฒนาซอฟต์แวร์นั้น
ส่วนมากไม่ได้เกิดจากอัจฉริยะที่เป็นผู้คิดค้นเลย
เรื่องพวกนั้นเป็นตำนานปรัมปราของแฮ็กเกอร์เสียมากกว่า

ผลลัพธ์ที่ออกมาร้อนแรงพอ ๆ  กัน จะว่าไปแล้ว
ก็เป็นความสำเร็จชนิดที่แฮ็กเกอร์ทุกคนฝันถึงเลย และนั่นหมายความว่า
ผมต้องตั้งมาตรฐานของตัวเองให้สูงขึ้น และในการทำให้ fetchmail
ดีได้อย่างที่ผมมองเห็นความเป็นไปได้นี้
นอกจากผมจะต้องเขียนเพื่อสนองความต้องการของตัวเองแล้ว
ยังต้องเพิ่มความสามารถที่คนอื่นที่อยู่นอกแวดวงของผมต้องการอีกด้วย
ในขณะเดียวกัน ก็ต้องรักษาตัวโปรแกรมให้เรียบง่ายและแน่นหนาดังเดิมด้วย

ความสามารถสำคัญอันแรกที่ผมเพิ่มเข้ามาหลังจากตระหนักถึงความจริงข้างต้น
คือการสนับสนุน multidrop
หรือความสามารถที่จะดึงเมลจากกล่องเมลรวมที่รวมเมลของผู้ใช้กลุ่มหนึ่ง
ไปกระจายให้กับผู้ใช้แต่ละคนตามที่จ่าหน้าในเมล

ผมตัดสินใจที่จะเพิ่มการสนับสนุน multidrop
ส่วนหนึ่งเพราะมีผู้ใช้จำนวนหนึ่งเรียกร้อง แต่สาเหตุสำคัญคือ
ผมคาดว่ามันจะช่วยแก้บั๊กต่าง ๆ  ในโค้ดส่วน single drop ออกไปด้วย
เพราะจะเป็นการบังคับให้ผมมาสนใจกับการจัดการเรื่องที่อยู่เมลในรูปทั่วไปจริง ๆ
เสียที และมันก็เป็นอย่างที่ผมคิด การแจงที่อยู่เมลแบบ
\href{http://info.internet.isi.edu:80/in-notes/rfc/files/rfc822.txt}{RFC
  822} ให้ถูกต้อง กินเวลาของผมไปมาก ไม่ใช่เพราะว่าโค้ดของมันยาก
แต่มันมีรายละเอียดที่เกี่ยวพันกันค่อนข้างเยอะ

แต่การเพิ่มการจัดการที่อยู่เมลแบบ multidrop เข้ามานั้น
กลายเป็นการตัดสินใจที่ยอดเยี่ยมเกี่ยวกับการออกแบบ ตอนนี้ผมรู้ว่า:

\begin{fancyquotes}
  14. เครื่องมือทั่วไปจะใช้ประโยชน์ได้ตามที่ตั้งใจไว้
  แต่เครื่องมือที่ยอดเยี่ยมจริง ๆ
  จะสามารถใช้ไปในทางที่ไม่ได้ตั้งใจไว้ได้ด้วย
\end{fancyquotes}

การใช้ fetchmail แบบ multidrop ที่คาดไม่ถึง คือใช้ทำเมลลิ่งลิสต์
โดยเก็บรายชื่อสมาชิกและกระจายเมลในฝั่ง {[}\emph{เครื่องลูกข่าย}{]}
ของการเชื่อมต่ออินเทอร์เน็ต ซึ่งหมายความว่า
ใครก็ตามที่มีเครื่องต่อกับอินเทอร์เน็ตผ่านบัญชีของ ISP
ก็สามารถจัดการกับเมลลิ่งลิสต์ได้ โดยไม่ต้องอาศัยแฟ้ม alias ในฝั่งของ ISP
เลย

การเปลี่ยนแปลงที่สำคัญอีกอย่างที่นักทดสอบของผมเรียกร้องคือ สนับสนุน MIME
(Multipurpose Internet Mail Extensions) แบบ 8 บิต
ซึ่งอันนี้ทำค่อนข้างง่าย เพราะผมได้ระวังให้โค้ดทำงานแบบ 8
บิตได้มาตั้งแต่ต้น (กล่าวคือ โดยไม่พยายามใช้งานบิตที่ 8
ที่ไม่ได้ใช้ในรหัส ASCII มาเก็บข้อมูลในโปรแกรม)
ที่ผมทำเช่นนี้ไม่ใช่เป็นเพราะคาดไว้ก่อนว่าจะต้องเพิ่มความสามารถนี้
แต่เป็นเพราะผมทำตามกฎอีกข้อหนึ่ง:

\begin{fancyquotes}
  15. เมื่อจะเขียนโปรแกรมที่เกี่ยวกับทางผ่านของข้อมูล (gateway) ใด ๆ
  พึงหลีกเลี่ยงการเปลี่ยนแปลงกระแสข้อมูลให้มากที่สุด และ {[}\emph{ห้าม}{]}
  ทิ้งข้อมูลทุกชนิด ยกเว้นผู้รับจะบังคับให้ทำเช่นนั้น!
\end{fancyquotes}

ถ้าผมไม่ปฏิบัติตามกฎนี้ การสนับสนุน MIME แบบ 8 บิตจะยากและเต็มไปด้วยบั๊ก
แต่สิ่งที่ผมต้องทำจริง ๆ  ก็แค่อ่านมาตรฐานของ MIME
(\href{http://info.internet.isi.edu:80/in-notes/rfc/files/rfc1652.txt}{RFC
  1652}) และเพิ่มส่วนแก้ไขข้อมูลส่วนหัวนิดเดียว

ผู้ใช้บางคนจากยุโรปเรียกร้องให้ผมเพิ่มตัวเลือกในการจำกัดจำนวนเมลที่จะดึงในแต่ละครั้ง
(เพื่อที่พวกเขาจะได้สามารถควบคุมค่าโทรศัพท์ได้)
ผมปฏิเสธเรื่องนี้เป็นเวลานาน และผมก็ยังไม่ชอบความสามารถนี้เท่าไรนัก
แต่ถ้าคุณเขียนโปรแกรมให้โลกใช้ คุณต้องฟังเสียงของผู้ใช้
เงื่อนไขนี้ไม่ได้เปลี่ยนแปลงเพียงเพราะเขาไม่ได้ได้จ่ายคุณเป็นตัวเงิน