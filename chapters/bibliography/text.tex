\chapter{บรรณานุกรม}

ผมยกคำพูดหลายแห่งมาจากหนังสืออมตะของ เฟรดเดอริก พี. บรูกส์ ชื่อ
\emph{The Mythical Man-Month}
เพราะแนวคิดของเขายังต้องการการพิสูจน์ต่อไปในหลาย ๆ  เรื่อง
ผมขอแนะนำอย่างยิ่ง ให้อ่านฉบับครบรอบ 25 ปีจาก Addison-Wesley (ISBN
0-201-83595-9) ซึ่งเพิ่มบทความ ``No Silver Bullet'' (ไม่มียาครอบจักรวาล)
ปี 1986 ของเขา

\noindent ฉบับปรับปรุงแก้ไขใหม่นี้ ยังปิดท้ายด้วยการหวนรำลึกเมื่อผ่านไป 20 ปี
อันเป็นบทที่ประเมินค่าไม่ได้ ในบทดังกล่าว
บรูกส์ยอมรับอย่างจริงใจเกี่ยวกับการตัดสินเล็กน้อยในเนื้อหาดั้งเดิมซึ่งไม่ผ่านการทดสอบของห้วงเวลา
ผมอ่านบทหวนรำลึกนี้ครั้งแรกหลังจากที่รุ่นแรกของบทความนี้เสร็จไปเยอะแล้ว
และต้องประหลาดใจที่ได้พบว่า
บรูกส์ได้ถือว่ากระบวนการที่คล้ายตลาดสดเป็นผลมาจากไมโครซอฟท์!
(อย่างไรก็ตาม ความจริงแล้ว การผูกโยงดังกล่าวกลายเป็นความผิดพลาด ในปี
1998 เราได้รู้จาก
\href{http://www.opensource.org/halloween/}{เอกสารวันฮัลโลวีน}
ว่าชุมชนนักพัฒนาภายในของไมโครซอฟท์นั้น แบ่งเป็นก๊กเป็นเหล่ามากมาย
ซึ่งการเข้าถึงซอร์สโดยทั่วไปที่จำเป็นสำหรับการทำงานแบบตลาดสดนั้น
ยังเป็นไปไม่ได้เลย)

\noindent  หนังสือของ เจอรัลด์ เอ็ม. เวนเบิร์ก ชื่อ \emph{The Psychology Of
  Computer Programming (จิตวิทยาของการเขียนโปรแกรมคอมพิวเตอร์)} (New York,
Van Nostrand Reinhold 1971) ได้เสนอแนวคิดที่ออกจะโชคร้ายที่ได้ชื่อว่า
``การเขียนโปรแกรมแบบไร้อัตตา''
ถึงแม้เขาจะไม่มีวี่แววว่าจะเป็นคนแรกที่ตระหนักถึงความสูญเปล่าของ
``หลักแห่งการบังคับบัญชา''
แต่เขาก็อาจเป็นคนแรกที่มองเห็นและโต้ประเด็นนี้
โดยเชื่อมโยงกับการพัฒนาซอฟต์แวร์โดยเฉพาะ

\noindent  ริชาร์ด พี. เกเบรียล
ซึ่งได้ตรึกตรองเกี่ยวกับวัฒนธรรมยูนิกซ์ก่อนยุคลินุกซ์
ได้โต้แย้งอย่างลังเล ถึงข้อได้เปรียบของรูปแบบคล้ายตลาดสดในบทความปี 1989
ชื่อ ``LISP: Good News, Bad News, and How To Win Big'' (LISP: ข่าวดี,
ข่าวร้าย, และวิธีชนะอย่างยิ่งใหญ่'') ของเขา แม้จะเก่าแล้วในบางเรื่อง
แต่บทความนี้ก็ยังเป็นที่ยกย่องในหมู่แฟน ๆ  ภาษา LISP (รวมถึงผมด้วย)
ผู้ร่วมแสดงความเห็นท่านหนึ่งเตือนผมว่า ตอนที่ชื่อ ``Worse Is Better''
(แย่กว่าดีกว่า) แทบจะเป็นการเก็งการเกิดของลินุกซ์ทีเดียว
บทความดังกล่าวสามารถอ่านในเว็บได้ที่
\url{http://www.naggum.no/worse-is-better.html}

\noindent  หนังสือของ เดอ มาร์โค และ ลิสเตอร์ ชื่อ \emph{Peopleware: Productive
  Projects and Teams (พีเพิลแวร์: โครงการและทีมงานอุดมผลงาน)} (New York;
Dorset House, 1987; ISBN 0-932633-05-6)
เป็นอัญมณีมีค่าที่ได้รับความชื่นชมน้อยกว่าที่ควร ซึ่งผมยินดีที่ได้เห็น
เฟรด บรูกส์ อ้างถึงในบทหวนรำลึกของเขา
แม้สิ่งที่ผู้เขียนกล่าวถึงจะเกี่ยวข้องโดยตรงกับชุมชนลินุกซ์หรือโอเพนซอร์สน้อยมาก
แต่แนวคิดของผู้เขียนเกี่ยวกับเงื่อนไขที่จำเป็นสำหรับงานสร้างสรรค์
ก็เป็นสิ่งที่เฉียบแหลม
และคุ้มค่าสำหรับใครก็ตามที่พยายามจะนำข้อดีของรูปแบบตลาดสดไปใช้ในบริบทเชิงพาณิชย์

\noindent  ท้ายที่สุด ผมต้องยอมรับว่า ผมเกือบจะเรียกบทความนี้ว่า ``The Cathedral
and the Agora'' จริง ๆ  โดยคำว่า agora นี้ เป็นภาษากรีก
ใช้เรียกตลาดเปิดโล่ง หรือที่ประชุมสาธารณะ บทความสัมมนาชื่อ ``agoric
systems'' ของ มาร์ค มิลเลอร์ และ เอริก เดร็กซ์เลอร์
ซึ่งได้บรรยายคุณสมบัติที่อุบัติขึ้นของระบบนิเวศน์คอมพิวเตอร์ที่คล้ายตลาด
ได้ช่วยให้ผมเตรียมพร้อมสำหรับการคิดอย่างชัดเจน
เกี่ยวกับปรากฏการณ์เทียบเคียงในวัฒนธรรมโอเพนซอร์ส
เมื่อลินุกซ์มากระตุ้นเตือนผมในห้าปีต่อมา บทความนี้อ่านได้บนเว็บที่
\url{http://www.agorics.com/agorpapers.html}