\chapter{ต้องส่งเมลให้ได้}

ตั้งแต่ปี 1993
ผมเป็นผู้บริหารงานด้านเทคนิคของผู้ให้บริการอินเทอร์เน็ตฟรีเล็ก ๆ
ที่ชื่อว่า Chester Country InterLink (CCIL) ซึ่งอยู่ใน West Chester
รัฐเพนซิลเวเนีย ผมเป็นผู้ร่วมก่อตั้ง CCIL
และได้เขียนซอฟต์แวร์กระดานข่าวที่ไม่เหมือนใครและรองรับหลายผู้ใช้ของเราขึ้น
ซึ่งคุณสามารถลองได้โดยเทลเน็ตไปยัง
\href{telnet://locke.ccil.org}{locke.ccil.org}
ปัจจุบันระบบนี้รองรับผู้ใช้สามพันคนด้วยสามสิบคู่สาย
งานนี้ทำให้ผมสามารถเข้าสู่อินเทอร์เน็ตได้ตลอด 24 ชั่วโมง
ผ่านเครือข่ายความเร็ว 56K ของ CCIL
ความจริงความสามารถนี้เป็นเรื่องจำเป็นในงานแบบนี้อยู่แล้ว

ผมเคยตัวกับการรับส่งอีเมลได้อย่างทันใจ จนรู้สึกว่าการต้องเทลเน็ตไปยัง
locke เป็นระยะ ๆ  เพื่อเช็กอีเมลนั้น เป็นเรื่องน่ารำคาญ
ผมต้องการให้อีเมลของผมถูกส่งไปยัง snark (ชื่อเครื่องที่บ้านผม)
เพื่อที่ผมจะได้รับการแจ้งเตือนเมื่ออีเมลมาถึง
และสามารถจัดการอีเมลด้วยโปรแกรมบนเครื่องของผมเอง

โพรโทคอลหลักสำหรับส่งเมลบนอินเทอร์เน็ต คือ SMTP (Simple Mail Tranfer
Protocol) นั้น ไม่ตรงกับความต้องการ
เพราะมันจะทำงานได้ดีต่อเมื่อเครื่องของเราต่ออินเทอร์เน็ตอยู่ตลอดเวลา
แต่เครื่องของผมไม่ได้ต่ออยู่ตลอด และไม่มีหมายเลขไอพีที่แน่นอนด้วย
สิ่งที่ผมต้องการคือโปรแกรมที่ติดต่อออกผ่านการเชื่อมต่อชนิดไม่ต่อเนื่อง
และดึงจดหมายมาส่งบนเครื่อง ผมรู้ว่ามีโปรแกรมประเภทนี้อยู่
และส่วนมากมักจะใช้โพรโทคอลแบบง่าย ๆ  ที่ชื่อ POP (Post Office Protocol)
ซึ่งโปรแกรมอีเมลทุกวันนี้ส่วนมากรู้จักและสนับสนุน แต่ว่าตอนนั้น
มันไม่มีอยู่ในโปรแกรมเมลที่ผมใช้อยู่

ผมต้องการโปรแกรมอ่าน POP3 ดังนั้นผมจึงค้นหาในอินเทอร์เน็ต
และพบโปรแกรมหนึ่ง ความจริงผมพบโปรแกรมอย่างนี้ถึง 3-4 ตัว
ผมลองใช้โปรแกรมหนึ่งไปสักพัก แต่มันขาดความสามารถที่ควรจะมีอย่างยิ่ง
คือการเข้าไปแก้ที่อยู่ของเมลที่ดึงมา เพื่อที่จะตอบจดหมายกลับได้ถูกต้อง

ปัญหามีดังนี้ สมมุติว่าใครบางคนชื่อ joe ที่อยู่ที่ locke ส่งเมลมาหาผม
ถ้าผมดึงเมลมายัง snark และตอบเมลฉบับนี้
โปรแกรมเมลของผมจะพยายามส่งไปยังผู้ใช้ที่ชื่อ joe บน snark ซึ่งไม่มีอยู่
และการแก้ที่อยู่เองให้เป็น \texttt{\textless{}@ccil.org\textgreater{}}
นั้น ก็ไม่ใช่เรื่องที่น่าสนุกนัก

สิ่งนี้เป็นสิ่งที่คอมพิวเตอร์ควรจะทำให้ผม แต่ว่าไม่มีโปรแกรมอ่าน POP
ตัวไหนเลยที่มีอยู่ในขณะนั้นที่ทำได้
และนี่ก็พาเรามารู้จักกับบทเรียนข้อแรก:

\begin{fancyquotes}
  1. ซอฟต์แวร์ดี ๆ  เริ่มมาจากการสนองความต้องการส่วนตัวของผู้พัฒนา
\end{fancyquotes}

เรื่องนี้คงชัดเจนอยู่แล้ว (มีสุภาษิตมานานแล้วว่า ``ความจำเป็น
คือบ่อเกิดของการคิดค้น'')
แต่ก็มีนักพัฒนาจำนวนมากที่ใช้เวลาแต่ละวันไปกับการปั่นงานแลกเงิน
เพื่อสร้างโปรแกรมที่เขาไม่ได้ต้องการ หรือไม่ได้รักที่จะทำ
แต่ไม่เป็นเช่นนั้นในโลกของลินุกซ์
ซึ่งอาจเป็นคำอธิบายว่าทำไมคุณภาพเฉลี่ยของซอฟต์แวร์ที่สร้างโดยชุมชนลินุกซ์จึงสูงกว่าปกติ

แล้วผมก็เลยกระโจนเข้าสู่การสร้างโปรแกรม POP3 ตัวใหม่อย่างบ้าคลั่ง
เพื่อเอาไปแข่งกับตัวเดิม ๆ  ทันทีอย่างนั้นหรือ? ไม่มีวันหรอก!
ผมได้สำรวจเครื่องมือจัดการ POP ที่มีอยู่ในมือ และถามตัวเองว่า
``โปรแกรมไหนที่ใกล้เคียงกับสิ่งที่ผมต้องการมากที่สุด?'' เพราะว่า:

\begin{fancyquotes}
  2. โปรแกรมเมอร์ที่ดีย่อมรู้ว่าจะเขียนอะไร
  แต่โปรแกรมเมอร์ที่ยอดเยี่ยมจะรู้ว่าอะไรต้องเขียนใหม่
  และอะไรใช้ของเก่าได้
\end{fancyquotes}

ผมไม่ได้บอกว่าตัวเองเป็นโปรแกรมเมอร์ที่ยอดเยี่ยม ผมแค่พยายามเลียนแบบดู
คุณสมบัติที่สำคัญของโปรแกรมเมอร์ที่ยอดเยี่ยมก็คือ
ความขี้เกียจอย่างสร้างสรรค์ พวกเขารู้ว่าคุณได้เกรด A
ไม่ใช่เพราะความพยายาม แต่เพราะผลของงานต่างหาก
และมันก็มักจะง่ายกว่าที่จะเริ่มต้นจากบางส่วนที่มีอยู่แล้ว
แทนที่จะเริ่มใหม่ทั้งหมด

ตัวอย่างเช่น \href{http://catb.org/~esr/faqs/linus}{ไลนัส ทอร์วัลด์}
ไม่ได้สร้างลินุกซ์ขึ้นมาจากศูนย์
เขาเริ่มด้วยโค้ดและความคิดบางส่วนจากมินิกซ์
ซึ่งเป็นยูนิกซ์ขนาดเล็กสำหรับเครื่องพีซี ต่อมาโค้ดของมินิกซ์ก็หายไปหมด
ถ้าไม่ถูกถอดออกก็ถูกแทนที่ด้วยโค้ดใหม่ ๆ  แต่ตอนที่ยังอยู่
มันได้ทำหน้าที่เสมือนโครงนั่งร้านสำหรับสร้างสรรค์ทารกที่จะกลายมาเป็นลินุกซ์ในที่สุด

ด้วยแนวคิดเดียวกัน ผมจึงค้นหาโปรแกรม POP ที่มีอยู่แล้ว
โดยเลือกตัวที่เขียนไว้อย่างดีพอสมควร เพื่อใช้เป็นจุดเริ่มต้นในการพัฒนา

ธรรมเนียมการแบ่งปันซอร์สโค้ดในโลกของยูนิกซ์
ทำให้การนำโค้ดไปใช้ใหม่เป็นเรื่องง่ายมาแต่ไหนแต่ไร (นี่เป็นเหตุผลว่าทำไม
GNU ถึงได้เลือกยูนิกซ์เป็นระบบปฏิบัติการหลัก
แม้จะสงวนท่าทีชัดเจนว่าไม่ใช่ยูนิกซ์ก็ตาม)
โลกของลินุกซ์ได้นำเอาธรรมเนียมนี้มาใช้อย่างเต็มพิกัดทางเทคโนโลยี
ทำให้เรามีโปรแกรมโอเพนซอร์สจำนวนมหาศาล
ดังนั้นการใช้เวลาเสาะหาโปรแกรมที่เกือบจะดีอยู่แล้วของใครสักคน
จะทำให้คุณได้ผลลัพธ์ดี ๆ  ในโลกของลินุกซ์มากกว่าที่อื่น

แล้วมันก็ได้ผลสำหรับผม เมื่อรวมกับโปรแกรมที่ผมพบครั้งก่อน
การค้นหาครั้งที่สองเพิ่มรายชื่อโปรแกรมที่เข้าท่าขึ้นเป็นเก้าตัว คือ
fetchpop, PopTart, get-mail, gwpop, pimp, pop-perl, popc, popmail และ
upop ตัวแรกที่ผมเลือกคือ fetchpop ซึ่งสร้างโดย Seung-Hong Oh
ผมได้เขียนความสามารถในการเปลี่ยนหัวจดหมายเข้าไป และปรับปรุงการทำงานอื่น ๆ
ซึ่งผู้สร้างได้รับเข้าไปใช้ในเวอร์ชัน 1.9

ไม่กี่สัปดาห์ต่อมา ผมบังเอิญได้อ่านโค้ดของ popclient ซึ่งสร้างโดย คาร์ล
แฮร์ริส เลยพบปัญหาว่า ถึงแม้ fetchpop จะมีแนวคิดดี ๆ  ที่ไม่เหมือนใคร
(อย่างเช่นการทำงานในโหมดดีมอนเบื้องหลัง) แต่มันทำงานได้กับ POP3 เท่านั้น
และโค้ดของมันก็ค่อนข้างเป็นงานของมือสมัครเล่น (ตอนนั้น Seung-Hong
เป็นโปรแกรมเมอร์ที่เก่ง แต่ขาดประสบการณ์
ซึ่งลักษณะทั้งสองแสดงให้เห็นในตัวโค้ด) ผมพบว่าโค้ดของคาร์ลดีกว่า
ดูเป็นมืออาชีพและแน่นหนา
แต่โปรแกรมของเขายังขาดความสามารถที่สำคัญและค่อนข้างทำยาก ที่มีใน
fetchpop (ซึ่งบางอย่างเป็นฝีมือของผม)

จะเปลี่ยนไหม? ถ้าผมเลือกจะเปลี่ยน ผมต้องทิ้งสิ่งที่ผมสร้างขึ้นมา
เพื่อแลกกับโปรแกรมใหม่อันเป็นฐานการพัฒนาที่ดีกว่า

แรงจูงใจในทางปฏิบัติที่จะเปลี่ยน
คือการได้ความสามารถในการสนับสนุนหลายโพรโทคอล โพรโทคอล POP3
เป็นโพรโทคอลที่ใช้กันมากในเครื่องแม่ข่ายตู้ไปรษณีย์เมล
แต่ก็ไม่ใช่โพรโทคอลเดียว fetchpop และโปรแกรมคู่แข่งตัวอื่นไม่สนับสนุน
POP2, RPOP หรือ APOP และผมยังมีเค้าความคิดที่จะเพิ่ม
\href{http://www.imap.org}{IMAP} (Internet Message Access Protocol)
ซึ่งเป็นโพรโทคอลที่ออกแบบมาใหม่ล่าสุด
และมีประสิทธิภาพมากที่สุดเข้าไปอีกด้วย เพื่อความมัน

แต่ผมมีเหตุผลทางทฤษฎีที่คิดว่าการเปลี่ยนอาจจะดีก็ได้
ซึ่งเป็นสิ่งที่ผมเรียนรู้มานานก่อนจะพบกับลินุกซ์

\begin{fancyquotes}
  3. ``เตรียมพร้อมที่จะทิ้งสิ่งเดิมไป คุณได้ทิ้งแน่ ไม่ว่าจะอย่างไร'' (จาก
  เฟรด บรูกส์ ใน \emph{The Mythical Man-Month}, บทที่ 11)
\end{fancyquotes}

หรืออีกนัยหนึ่ง คุณมักจะไม่ได้เข้าใจปัญหาอย่างแท้จริง
จนกว่าคุณจะเริ่มลงมือทำครั้งแรก พอลงมือครั้งที่สอง
คุณอาจรู้มากพอจะทำสิ่งที่ถูกแล้ว ดังนั้นถ้าคุณต้องการทำให้ถูกต้องจริง ๆ
ก็ควรเตรียมพร้อมที่จะเริ่มต้นใหม่ \pagenote{ในหนังสือ \emph{Programing Pearls
    (อัญมณีแห่งการเขียนโปรแกรม)} จอน เบนทลีย์
  นักตั้งคติพจน์ทางวิทยาการคอมพิวเตอร์
  ได้ให้ความเห็นต่อข้อสังเกตของบรูกส์ไว้ว่า
  ``ถ้าคุณเตรียมพร้อมที่จะทิ้งสิ่งหนึ่ง คุณจะได้ทิ้งไปสองสิ่ง''
  เขากล่าวได้ถูกต้องค่อนข้างแน่นอนทีเดียว
  ประเด็นของข้อสังเกตของบรูกส์และเบนทลีย์
  ไม่ใช่เพียงแค่ว่าคุณควรคาดได้ว่าความพยายามครั้งแรกจะผิดเท่านั้น
  แต่ยังย้ำว่า การตั้งต้นใหม่ด้วยแนวคิดที่ถูกต้อง
  มักจะเกิดผลมากกว่าการพยายามกู้ซากที่เละเทะ}

ผมบอกตัวเองว่าสิ่งที่ผมเพิ่มเข้าไปใน fetchpop
เป็นความพยายามครั้งแรกของผม ดังนั้นผมจึงเปลี่ยน

หลังจากผมส่งแพตช์ชุดแรกไปให้ คาร์ล แฮร์ริส เมื่อวันที่ 25 มิถุนายน 1996
ผมพบว่าเขาเลิกสนใจ popclient ไปก่อนหน้านั้นแล้ว ตัวโค้ดดูสกปรก
และมีบั๊กเล็ก ๆ  น้อย ๆ  ประปราย ผมมีเรื่องที่อยากแก้หลายจุด
และเราก็ตกลงกันได้อย่างรวดเร็ว
ว่าสิ่งที่สมเหตุสมผลที่สุดคือผมควรจะรับโปรแกรมนี้ไปดูแลต่อ

เผลอไม่ทันไร โครงการของผมก็โตพรวดพราด
ผมไม่ได้คิดถึงแค่แพตช์เล็กแพตช์น้อยสำหรับโปรแกรม POP
ที่มีอยู่แล้วอีกต่อไป ผมกำลังดูแลโปรแกรมทั้งตัว
และผมก็เกิดความคิดในสมองมากมาย
ซึ่งผมรู้ดีว่าจะนำไปสู่ความเปลี่ยนแปลงขนานใหญ่

ในวัฒนธรรมซอฟต์แวร์ที่เน้นการแบ่งปันโค้ด
นี่เป็นวิถีทางตามธรรมชาติที่โครงการต่าง ๆ  จะก้าวหน้าต่อไป
ผมกำลังทำตามหลักการนี้:

\begin{fancyquotes}
  4. ถ้าคุณมีทัศนคติที่เหมาะสม ปัญหาที่น่าสนใจจะมาหาคุณเอง
\end{fancyquotes}

แต่ทัศนคติของ คาร์ล แฮร์ริส นั้นสำคัญยิ่งกว่า เขาเข้าใจดีว่า

\begin{fancyquotes}
  5. เมื่อคุณหมดความสนใจในโปรแกรมใดแล้ว หน้าที่สุดท้ายของคุณคือ
  ส่งมันต่อให้กับผู้สืบทอดที่มีความสามารถ
\end{fancyquotes}

ผมและคาร์ลรู้โดยไม่ต้องคุยกันในเรื่องนี้เลย
เรามีเป้าหมายร่วมกันที่จะหาคำตอบที่ดีที่สุดที่มีอยู่
คำถามเดียวที่เกิดขึ้นกับแต่ละฝ่ายก็คือ ผมจะพิสูจน์ได้ไหม
ว่าผมคือผู้ดูแลที่ควรวางใจ เมื่อผมทำได้ เขาก็จากไปอย่างนุ่มนวล
ผมหวังว่าผมจะทำเช่นนั้นเหมือนกัน เมื่อถึงตาที่ผมต้องส่งต่อให้คนอื่น